\documentclass[11pt]{article}

    \usepackage[breakable]{tcolorbox}
    \usepackage{parskip} % Stop auto-indenting (to mimic markdown behaviour)
    
    \usepackage{iftex}
    \ifPDFTeX
    	\usepackage[T1]{fontenc}
    	\usepackage{mathpazo}
    \else
    	\usepackage{fontspec}
    \fi

    % Basic figure setup, for now with no caption control since it's done
    % automatically by Pandoc (which extracts ![](path) syntax from Markdown).
    \usepackage{graphicx}
    % Maintain compatibility with old templates. Remove in nbconvert 6.0
    \let\Oldincludegraphics\includegraphics
    % Ensure that by default, figures have no caption (until we provide a
    % proper Figure object with a Caption API and a way to capture that
    % in the conversion process - todo).
    \usepackage{caption}
    \DeclareCaptionFormat{nocaption}{}
    \captionsetup{format=nocaption,aboveskip=0pt,belowskip=0pt}

    \usepackage{float}
    \floatplacement{figure}{H} % forces figures to be placed at the correct location
    \usepackage{xcolor} % Allow colors to be defined
    \usepackage{enumerate} % Needed for markdown enumerations to work
    \usepackage{geometry} % Used to adjust the document margins
    \usepackage{amsmath} % Equations
    \usepackage{amssymb} % Equations
    \usepackage{textcomp} % defines textquotesingle
    % Hack from http://tex.stackexchange.com/a/47451/13684:
    \AtBeginDocument{%
        \def\PYZsq{\textquotesingle}% Upright quotes in Pygmentized code
    }
    \usepackage{upquote} % Upright quotes for verbatim code
    \usepackage{eurosym} % defines \euro
    \usepackage[mathletters]{ucs} % Extended unicode (utf-8) support
    \usepackage{fancyvrb} % verbatim replacement that allows latex
    \usepackage{grffile} % extends the file name processing of package graphics 
                         % to support a larger range
    \makeatletter % fix for old versions of grffile with XeLaTeX
    \@ifpackagelater{grffile}{2019/11/01}
    {
      % Do nothing on new versions
    }
    {
      \def\Gread@@xetex#1{%
        \IfFileExists{"\Gin@base".bb}%
        {\Gread@eps{\Gin@base.bb}}%
        {\Gread@@xetex@aux#1}%
      }
    }
    \makeatother
    \usepackage[Export]{adjustbox} % Used to constrain images to a maximum size
    \adjustboxset{max size={0.9\linewidth}{0.9\paperheight}}

    % The hyperref package gives us a pdf with properly built
    % internal navigation ('pdf bookmarks' for the table of contents,
    % internal cross-reference links, web links for URLs, etc.)
    \usepackage{hyperref}
    % The default LaTeX title has an obnoxious amount of whitespace. By default,
    % titling removes some of it. It also provides customization options.
    \usepackage{titling}
    \usepackage{longtable} % longtable support required by pandoc >1.10
    \usepackage{booktabs}  % table support for pandoc > 1.12.2
    \usepackage[inline]{enumitem} % IRkernel/repr support (it uses the enumerate* environment)
    \usepackage[normalem]{ulem} % ulem is needed to support strikethroughs (\sout)
                                % normalem makes italics be italics, not underlines
    \usepackage{mathrsfs}
    

    
    % Colors for the hyperref package
    \definecolor{urlcolor}{rgb}{0,.145,.698}
    \definecolor{linkcolor}{rgb}{.71,0.21,0.01}
    \definecolor{citecolor}{rgb}{.12,.54,.11}

    % ANSI colors
    \definecolor{ansi-black}{HTML}{3E424D}
    \definecolor{ansi-black-intense}{HTML}{282C36}
    \definecolor{ansi-red}{HTML}{E75C58}
    \definecolor{ansi-red-intense}{HTML}{B22B31}
    \definecolor{ansi-green}{HTML}{00A250}
    \definecolor{ansi-green-intense}{HTML}{007427}
    \definecolor{ansi-yellow}{HTML}{DDB62B}
    \definecolor{ansi-yellow-intense}{HTML}{B27D12}
    \definecolor{ansi-blue}{HTML}{208FFB}
    \definecolor{ansi-blue-intense}{HTML}{0065CA}
    \definecolor{ansi-magenta}{HTML}{D160C4}
    \definecolor{ansi-magenta-intense}{HTML}{A03196}
    \definecolor{ansi-cyan}{HTML}{60C6C8}
    \definecolor{ansi-cyan-intense}{HTML}{258F8F}
    \definecolor{ansi-white}{HTML}{C5C1B4}
    \definecolor{ansi-white-intense}{HTML}{A1A6B2}
    \definecolor{ansi-default-inverse-fg}{HTML}{FFFFFF}
    \definecolor{ansi-default-inverse-bg}{HTML}{000000}

    % common color for the border for error outputs.
    \definecolor{outerrorbackground}{HTML}{FFDFDF}

    % commands and environments needed by pandoc snippets
    % extracted from the output of `pandoc -s`
    \providecommand{\tightlist}{%
      \setlength{\itemsep}{0pt}\setlength{\parskip}{0pt}}
    \DefineVerbatimEnvironment{Highlighting}{Verbatim}{commandchars=\\\{\}}
    % Add ',fontsize=\small' for more characters per line
    \newenvironment{Shaded}{}{}
    \newcommand{\KeywordTok}[1]{\textcolor[rgb]{0.00,0.44,0.13}{\textbf{{#1}}}}
    \newcommand{\DataTypeTok}[1]{\textcolor[rgb]{0.56,0.13,0.00}{{#1}}}
    \newcommand{\DecValTok}[1]{\textcolor[rgb]{0.25,0.63,0.44}{{#1}}}
    \newcommand{\BaseNTok}[1]{\textcolor[rgb]{0.25,0.63,0.44}{{#1}}}
    \newcommand{\FloatTok}[1]{\textcolor[rgb]{0.25,0.63,0.44}{{#1}}}
    \newcommand{\CharTok}[1]{\textcolor[rgb]{0.25,0.44,0.63}{{#1}}}
    \newcommand{\StringTok}[1]{\textcolor[rgb]{0.25,0.44,0.63}{{#1}}}
    \newcommand{\CommentTok}[1]{\textcolor[rgb]{0.38,0.63,0.69}{\textit{{#1}}}}
    \newcommand{\OtherTok}[1]{\textcolor[rgb]{0.00,0.44,0.13}{{#1}}}
    \newcommand{\AlertTok}[1]{\textcolor[rgb]{1.00,0.00,0.00}{\textbf{{#1}}}}
    \newcommand{\FunctionTok}[1]{\textcolor[rgb]{0.02,0.16,0.49}{{#1}}}
    \newcommand{\RegionMarkerTok}[1]{{#1}}
    \newcommand{\ErrorTok}[1]{\textcolor[rgb]{1.00,0.00,0.00}{\textbf{{#1}}}}
    \newcommand{\NormalTok}[1]{{#1}}
    
    % Additional commands for more recent versions of Pandoc
    \newcommand{\ConstantTok}[1]{\textcolor[rgb]{0.53,0.00,0.00}{{#1}}}
    \newcommand{\SpecialCharTok}[1]{\textcolor[rgb]{0.25,0.44,0.63}{{#1}}}
    \newcommand{\VerbatimStringTok}[1]{\textcolor[rgb]{0.25,0.44,0.63}{{#1}}}
    \newcommand{\SpecialStringTok}[1]{\textcolor[rgb]{0.73,0.40,0.53}{{#1}}}
    \newcommand{\ImportTok}[1]{{#1}}
    \newcommand{\DocumentationTok}[1]{\textcolor[rgb]{0.73,0.13,0.13}{\textit{{#1}}}}
    \newcommand{\AnnotationTok}[1]{\textcolor[rgb]{0.38,0.63,0.69}{\textbf{\textit{{#1}}}}}
    \newcommand{\CommentVarTok}[1]{\textcolor[rgb]{0.38,0.63,0.69}{\textbf{\textit{{#1}}}}}
    \newcommand{\VariableTok}[1]{\textcolor[rgb]{0.10,0.09,0.49}{{#1}}}
    \newcommand{\ControlFlowTok}[1]{\textcolor[rgb]{0.00,0.44,0.13}{\textbf{{#1}}}}
    \newcommand{\OperatorTok}[1]{\textcolor[rgb]{0.40,0.40,0.40}{{#1}}}
    \newcommand{\BuiltInTok}[1]{{#1}}
    \newcommand{\ExtensionTok}[1]{{#1}}
    \newcommand{\PreprocessorTok}[1]{\textcolor[rgb]{0.74,0.48,0.00}{{#1}}}
    \newcommand{\AttributeTok}[1]{\textcolor[rgb]{0.49,0.56,0.16}{{#1}}}
    \newcommand{\InformationTok}[1]{\textcolor[rgb]{0.38,0.63,0.69}{\textbf{\textit{{#1}}}}}
    \newcommand{\WarningTok}[1]{\textcolor[rgb]{0.38,0.63,0.69}{\textbf{\textit{{#1}}}}}
    
    
    % Define a nice break command that doesn't care if a line doesn't already
    % exist.
    \def\br{\hspace*{\fill} \\* }
    % Math Jax compatibility definitions
    \def\gt{>}
    \def\lt{<}
    \let\Oldtex\TeX
    \let\Oldlatex\LaTeX
    \renewcommand{\TeX}{\textrm{\Oldtex}}
    \renewcommand{\LaTeX}{\textrm{\Oldlatex}}
    % Document parameters
    % Document title
    \title{Take-Home-test-Nu-Roberto}
    
    
    
    
    
% Pygments definitions
\makeatletter
\def\PY@reset{\let\PY@it=\relax \let\PY@bf=\relax%
    \let\PY@ul=\relax \let\PY@tc=\relax%
    \let\PY@bc=\relax \let\PY@ff=\relax}
\def\PY@tok#1{\csname PY@tok@#1\endcsname}
\def\PY@toks#1+{\ifx\relax#1\empty\else%
    \PY@tok{#1}\expandafter\PY@toks\fi}
\def\PY@do#1{\PY@bc{\PY@tc{\PY@ul{%
    \PY@it{\PY@bf{\PY@ff{#1}}}}}}}
\def\PY#1#2{\PY@reset\PY@toks#1+\relax+\PY@do{#2}}

\@namedef{PY@tok@w}{\def\PY@tc##1{\textcolor[rgb]{0.73,0.73,0.73}{##1}}}
\@namedef{PY@tok@c}{\let\PY@it=\textit\def\PY@tc##1{\textcolor[rgb]{0.25,0.50,0.50}{##1}}}
\@namedef{PY@tok@cp}{\def\PY@tc##1{\textcolor[rgb]{0.74,0.48,0.00}{##1}}}
\@namedef{PY@tok@k}{\let\PY@bf=\textbf\def\PY@tc##1{\textcolor[rgb]{0.00,0.50,0.00}{##1}}}
\@namedef{PY@tok@kp}{\def\PY@tc##1{\textcolor[rgb]{0.00,0.50,0.00}{##1}}}
\@namedef{PY@tok@kt}{\def\PY@tc##1{\textcolor[rgb]{0.69,0.00,0.25}{##1}}}
\@namedef{PY@tok@o}{\def\PY@tc##1{\textcolor[rgb]{0.40,0.40,0.40}{##1}}}
\@namedef{PY@tok@ow}{\let\PY@bf=\textbf\def\PY@tc##1{\textcolor[rgb]{0.67,0.13,1.00}{##1}}}
\@namedef{PY@tok@nb}{\def\PY@tc##1{\textcolor[rgb]{0.00,0.50,0.00}{##1}}}
\@namedef{PY@tok@nf}{\def\PY@tc##1{\textcolor[rgb]{0.00,0.00,1.00}{##1}}}
\@namedef{PY@tok@nc}{\let\PY@bf=\textbf\def\PY@tc##1{\textcolor[rgb]{0.00,0.00,1.00}{##1}}}
\@namedef{PY@tok@nn}{\let\PY@bf=\textbf\def\PY@tc##1{\textcolor[rgb]{0.00,0.00,1.00}{##1}}}
\@namedef{PY@tok@ne}{\let\PY@bf=\textbf\def\PY@tc##1{\textcolor[rgb]{0.82,0.25,0.23}{##1}}}
\@namedef{PY@tok@nv}{\def\PY@tc##1{\textcolor[rgb]{0.10,0.09,0.49}{##1}}}
\@namedef{PY@tok@no}{\def\PY@tc##1{\textcolor[rgb]{0.53,0.00,0.00}{##1}}}
\@namedef{PY@tok@nl}{\def\PY@tc##1{\textcolor[rgb]{0.63,0.63,0.00}{##1}}}
\@namedef{PY@tok@ni}{\let\PY@bf=\textbf\def\PY@tc##1{\textcolor[rgb]{0.60,0.60,0.60}{##1}}}
\@namedef{PY@tok@na}{\def\PY@tc##1{\textcolor[rgb]{0.49,0.56,0.16}{##1}}}
\@namedef{PY@tok@nt}{\let\PY@bf=\textbf\def\PY@tc##1{\textcolor[rgb]{0.00,0.50,0.00}{##1}}}
\@namedef{PY@tok@nd}{\def\PY@tc##1{\textcolor[rgb]{0.67,0.13,1.00}{##1}}}
\@namedef{PY@tok@s}{\def\PY@tc##1{\textcolor[rgb]{0.73,0.13,0.13}{##1}}}
\@namedef{PY@tok@sd}{\let\PY@it=\textit\def\PY@tc##1{\textcolor[rgb]{0.73,0.13,0.13}{##1}}}
\@namedef{PY@tok@si}{\let\PY@bf=\textbf\def\PY@tc##1{\textcolor[rgb]{0.73,0.40,0.53}{##1}}}
\@namedef{PY@tok@se}{\let\PY@bf=\textbf\def\PY@tc##1{\textcolor[rgb]{0.73,0.40,0.13}{##1}}}
\@namedef{PY@tok@sr}{\def\PY@tc##1{\textcolor[rgb]{0.73,0.40,0.53}{##1}}}
\@namedef{PY@tok@ss}{\def\PY@tc##1{\textcolor[rgb]{0.10,0.09,0.49}{##1}}}
\@namedef{PY@tok@sx}{\def\PY@tc##1{\textcolor[rgb]{0.00,0.50,0.00}{##1}}}
\@namedef{PY@tok@m}{\def\PY@tc##1{\textcolor[rgb]{0.40,0.40,0.40}{##1}}}
\@namedef{PY@tok@gh}{\let\PY@bf=\textbf\def\PY@tc##1{\textcolor[rgb]{0.00,0.00,0.50}{##1}}}
\@namedef{PY@tok@gu}{\let\PY@bf=\textbf\def\PY@tc##1{\textcolor[rgb]{0.50,0.00,0.50}{##1}}}
\@namedef{PY@tok@gd}{\def\PY@tc##1{\textcolor[rgb]{0.63,0.00,0.00}{##1}}}
\@namedef{PY@tok@gi}{\def\PY@tc##1{\textcolor[rgb]{0.00,0.63,0.00}{##1}}}
\@namedef{PY@tok@gr}{\def\PY@tc##1{\textcolor[rgb]{1.00,0.00,0.00}{##1}}}
\@namedef{PY@tok@ge}{\let\PY@it=\textit}
\@namedef{PY@tok@gs}{\let\PY@bf=\textbf}
\@namedef{PY@tok@gp}{\let\PY@bf=\textbf\def\PY@tc##1{\textcolor[rgb]{0.00,0.00,0.50}{##1}}}
\@namedef{PY@tok@go}{\def\PY@tc##1{\textcolor[rgb]{0.53,0.53,0.53}{##1}}}
\@namedef{PY@tok@gt}{\def\PY@tc##1{\textcolor[rgb]{0.00,0.27,0.87}{##1}}}
\@namedef{PY@tok@err}{\def\PY@bc##1{{\setlength{\fboxsep}{\string -\fboxrule}\fcolorbox[rgb]{1.00,0.00,0.00}{1,1,1}{\strut ##1}}}}
\@namedef{PY@tok@kc}{\let\PY@bf=\textbf\def\PY@tc##1{\textcolor[rgb]{0.00,0.50,0.00}{##1}}}
\@namedef{PY@tok@kd}{\let\PY@bf=\textbf\def\PY@tc##1{\textcolor[rgb]{0.00,0.50,0.00}{##1}}}
\@namedef{PY@tok@kn}{\let\PY@bf=\textbf\def\PY@tc##1{\textcolor[rgb]{0.00,0.50,0.00}{##1}}}
\@namedef{PY@tok@kr}{\let\PY@bf=\textbf\def\PY@tc##1{\textcolor[rgb]{0.00,0.50,0.00}{##1}}}
\@namedef{PY@tok@bp}{\def\PY@tc##1{\textcolor[rgb]{0.00,0.50,0.00}{##1}}}
\@namedef{PY@tok@fm}{\def\PY@tc##1{\textcolor[rgb]{0.00,0.00,1.00}{##1}}}
\@namedef{PY@tok@vc}{\def\PY@tc##1{\textcolor[rgb]{0.10,0.09,0.49}{##1}}}
\@namedef{PY@tok@vg}{\def\PY@tc##1{\textcolor[rgb]{0.10,0.09,0.49}{##1}}}
\@namedef{PY@tok@vi}{\def\PY@tc##1{\textcolor[rgb]{0.10,0.09,0.49}{##1}}}
\@namedef{PY@tok@vm}{\def\PY@tc##1{\textcolor[rgb]{0.10,0.09,0.49}{##1}}}
\@namedef{PY@tok@sa}{\def\PY@tc##1{\textcolor[rgb]{0.73,0.13,0.13}{##1}}}
\@namedef{PY@tok@sb}{\def\PY@tc##1{\textcolor[rgb]{0.73,0.13,0.13}{##1}}}
\@namedef{PY@tok@sc}{\def\PY@tc##1{\textcolor[rgb]{0.73,0.13,0.13}{##1}}}
\@namedef{PY@tok@dl}{\def\PY@tc##1{\textcolor[rgb]{0.73,0.13,0.13}{##1}}}
\@namedef{PY@tok@s2}{\def\PY@tc##1{\textcolor[rgb]{0.73,0.13,0.13}{##1}}}
\@namedef{PY@tok@sh}{\def\PY@tc##1{\textcolor[rgb]{0.73,0.13,0.13}{##1}}}
\@namedef{PY@tok@s1}{\def\PY@tc##1{\textcolor[rgb]{0.73,0.13,0.13}{##1}}}
\@namedef{PY@tok@mb}{\def\PY@tc##1{\textcolor[rgb]{0.40,0.40,0.40}{##1}}}
\@namedef{PY@tok@mf}{\def\PY@tc##1{\textcolor[rgb]{0.40,0.40,0.40}{##1}}}
\@namedef{PY@tok@mh}{\def\PY@tc##1{\textcolor[rgb]{0.40,0.40,0.40}{##1}}}
\@namedef{PY@tok@mi}{\def\PY@tc##1{\textcolor[rgb]{0.40,0.40,0.40}{##1}}}
\@namedef{PY@tok@il}{\def\PY@tc##1{\textcolor[rgb]{0.40,0.40,0.40}{##1}}}
\@namedef{PY@tok@mo}{\def\PY@tc##1{\textcolor[rgb]{0.40,0.40,0.40}{##1}}}
\@namedef{PY@tok@ch}{\let\PY@it=\textit\def\PY@tc##1{\textcolor[rgb]{0.25,0.50,0.50}{##1}}}
\@namedef{PY@tok@cm}{\let\PY@it=\textit\def\PY@tc##1{\textcolor[rgb]{0.25,0.50,0.50}{##1}}}
\@namedef{PY@tok@cpf}{\let\PY@it=\textit\def\PY@tc##1{\textcolor[rgb]{0.25,0.50,0.50}{##1}}}
\@namedef{PY@tok@c1}{\let\PY@it=\textit\def\PY@tc##1{\textcolor[rgb]{0.25,0.50,0.50}{##1}}}
\@namedef{PY@tok@cs}{\let\PY@it=\textit\def\PY@tc##1{\textcolor[rgb]{0.25,0.50,0.50}{##1}}}

\def\PYZbs{\char`\\}
\def\PYZus{\char`\_}
\def\PYZob{\char`\{}
\def\PYZcb{\char`\}}
\def\PYZca{\char`\^}
\def\PYZam{\char`\&}
\def\PYZlt{\char`\<}
\def\PYZgt{\char`\>}
\def\PYZsh{\char`\#}
\def\PYZpc{\char`\%}
\def\PYZdl{\char`\$}
\def\PYZhy{\char`\-}
\def\PYZsq{\char`\'}
\def\PYZdq{\char`\"}
\def\PYZti{\char`\~}
% for compatibility with earlier versions
\def\PYZat{@}
\def\PYZlb{[}
\def\PYZrb{]}
\makeatother


    % For linebreaks inside Verbatim environment from package fancyvrb. 
    \makeatletter
        \newbox\Wrappedcontinuationbox 
        \newbox\Wrappedvisiblespacebox 
        \newcommand*\Wrappedvisiblespace {\textcolor{red}{\textvisiblespace}} 
        \newcommand*\Wrappedcontinuationsymbol {\textcolor{red}{\llap{\tiny$\m@th\hookrightarrow$}}} 
        \newcommand*\Wrappedcontinuationindent {3ex } 
        \newcommand*\Wrappedafterbreak {\kern\Wrappedcontinuationindent\copy\Wrappedcontinuationbox} 
        % Take advantage of the already applied Pygments mark-up to insert 
        % potential linebreaks for TeX processing. 
        %        {, <, #, %, $, ' and ": go to next line. 
        %        _, }, ^, &, >, - and ~: stay at end of broken line. 
        % Use of \textquotesingle for straight quote. 
        \newcommand*\Wrappedbreaksatspecials {% 
            \def\PYGZus{\discretionary{\char`\_}{\Wrappedafterbreak}{\char`\_}}% 
            \def\PYGZob{\discretionary{}{\Wrappedafterbreak\char`\{}{\char`\{}}% 
            \def\PYGZcb{\discretionary{\char`\}}{\Wrappedafterbreak}{\char`\}}}% 
            \def\PYGZca{\discretionary{\char`\^}{\Wrappedafterbreak}{\char`\^}}% 
            \def\PYGZam{\discretionary{\char`\&}{\Wrappedafterbreak}{\char`\&}}% 
            \def\PYGZlt{\discretionary{}{\Wrappedafterbreak\char`\<}{\char`\<}}% 
            \def\PYGZgt{\discretionary{\char`\>}{\Wrappedafterbreak}{\char`\>}}% 
            \def\PYGZsh{\discretionary{}{\Wrappedafterbreak\char`\#}{\char`\#}}% 
            \def\PYGZpc{\discretionary{}{\Wrappedafterbreak\char`\%}{\char`\%}}% 
            \def\PYGZdl{\discretionary{}{\Wrappedafterbreak\char`\$}{\char`\$}}% 
            \def\PYGZhy{\discretionary{\char`\-}{\Wrappedafterbreak}{\char`\-}}% 
            \def\PYGZsq{\discretionary{}{\Wrappedafterbreak\textquotesingle}{\textquotesingle}}% 
            \def\PYGZdq{\discretionary{}{\Wrappedafterbreak\char`\"}{\char`\"}}% 
            \def\PYGZti{\discretionary{\char`\~}{\Wrappedafterbreak}{\char`\~}}% 
        } 
        % Some characters . , ; ? ! / are not pygmentized. 
        % This macro makes them "active" and they will insert potential linebreaks 
        \newcommand*\Wrappedbreaksatpunct {% 
            \lccode`\~`\.\lowercase{\def~}{\discretionary{\hbox{\char`\.}}{\Wrappedafterbreak}{\hbox{\char`\.}}}% 
            \lccode`\~`\,\lowercase{\def~}{\discretionary{\hbox{\char`\,}}{\Wrappedafterbreak}{\hbox{\char`\,}}}% 
            \lccode`\~`\;\lowercase{\def~}{\discretionary{\hbox{\char`\;}}{\Wrappedafterbreak}{\hbox{\char`\;}}}% 
            \lccode`\~`\:\lowercase{\def~}{\discretionary{\hbox{\char`\:}}{\Wrappedafterbreak}{\hbox{\char`\:}}}% 
            \lccode`\~`\?\lowercase{\def~}{\discretionary{\hbox{\char`\?}}{\Wrappedafterbreak}{\hbox{\char`\?}}}% 
            \lccode`\~`\!\lowercase{\def~}{\discretionary{\hbox{\char`\!}}{\Wrappedafterbreak}{\hbox{\char`\!}}}% 
            \lccode`\~`\/\lowercase{\def~}{\discretionary{\hbox{\char`\/}}{\Wrappedafterbreak}{\hbox{\char`\/}}}% 
            \catcode`\.\active
            \catcode`\,\active 
            \catcode`\;\active
            \catcode`\:\active
            \catcode`\?\active
            \catcode`\!\active
            \catcode`\/\active 
            \lccode`\~`\~ 	
        }
    \makeatother

    \let\OriginalVerbatim=\Verbatim
    \makeatletter
    \renewcommand{\Verbatim}[1][1]{%
        %\parskip\z@skip
        \sbox\Wrappedcontinuationbox {\Wrappedcontinuationsymbol}%
        \sbox\Wrappedvisiblespacebox {\FV@SetupFont\Wrappedvisiblespace}%
        \def\FancyVerbFormatLine ##1{\hsize\linewidth
            \vtop{\raggedright\hyphenpenalty\z@\exhyphenpenalty\z@
                \doublehyphendemerits\z@\finalhyphendemerits\z@
                \strut ##1\strut}%
        }%
        % If the linebreak is at a space, the latter will be displayed as visible
        % space at end of first line, and a continuation symbol starts next line.
        % Stretch/shrink are however usually zero for typewriter font.
        \def\FV@Space {%
            \nobreak\hskip\z@ plus\fontdimen3\font minus\fontdimen4\font
            \discretionary{\copy\Wrappedvisiblespacebox}{\Wrappedafterbreak}
            {\kern\fontdimen2\font}%
        }%
        
        % Allow breaks at special characters using \PYG... macros.
        \Wrappedbreaksatspecials
        % Breaks at punctuation characters . , ; ? ! and / need catcode=\active 	
        \OriginalVerbatim[#1,codes*=\Wrappedbreaksatpunct]%
    }
    \makeatother

    % Exact colors from NB
    \definecolor{incolor}{HTML}{303F9F}
    \definecolor{outcolor}{HTML}{D84315}
    \definecolor{cellborder}{HTML}{CFCFCF}
    \definecolor{cellbackground}{HTML}{F7F7F7}
    
    % prompt
    \makeatletter
    \newcommand{\boxspacing}{\kern\kvtcb@left@rule\kern\kvtcb@boxsep}
    \makeatother
    \newcommand{\prompt}[4]{
        {\ttfamily\llap{{\color{#2}[#3]:\hspace{3pt}#4}}\vspace{-\baselineskip}}
    }
    

    
    % Prevent overflowing lines due to hard-to-break entities
    \sloppy 
    % Setup hyperref package
    \hypersetup{
      breaklinks=true,  % so long urls are correctly broken across lines
      colorlinks=true,
      urlcolor=urlcolor,
      linkcolor=linkcolor,
      citecolor=citecolor,
      }
    % Slightly bigger margins than the latex defaults
    
    \geometry{verbose,tmargin=1in,bmargin=1in,lmargin=1in,rmargin=1in}
    
    

\begin{document}
    \title{Take-Home test 1Q22}
    \author{Roberto Vásquez Martínez}
    \date{\today}
    \maketitle

    
    \hypertarget{take-home-test-nu-mx-2022}{%
\section{Take Home test Nu MX 2022}\label{take-home-test-nu-mx-2022}}

    \hypertarget{data-wrangling-exploration}{%
\subsection{Data Wrangling +
Exploration}\label{data-wrangling-exploration}}

    In the next code cell, we charge the datasets that help us in the first
section. We wil use \texttt{pandas} Python library in order to
manipulate the datasets.

    \begin{tcolorbox}[breakable, size=fbox, boxrule=1pt, pad at break*=1mm,colback=cellbackground, colframe=cellborder]
\prompt{In}{incolor}{1}{\boxspacing}
\begin{Verbatim}[commandchars=\\\{\}]
\PY{k+kn}{import} \PY{n+nn}{pandas} \PY{k}{as} \PY{n+nn}{pd}
\PY{n}{magic\PYZus{}towns}\PY{o}{=}\PY{n}{pd}\PY{o}{.}\PY{n}{read\PYZus{}csv}\PY{p}{(}\PY{l+s+s1}{\PYZsq{}}\PY{l+s+s1}{pueblos\PYZus{}magicos.csv}\PY{l+s+s1}{\PYZsq{}}\PY{p}{)}
\PY{n}{mexico\PYZus{}tourism}\PY{o}{=}\PY{n}{pd}\PY{o}{.}\PY{n}{read\PYZus{}csv}\PY{p}{(}\PY{l+s+s1}{\PYZsq{}}\PY{l+s+s1}{turismo\PYZus{}mexico.csv}\PY{l+s+s1}{\PYZsq{}}\PY{p}{)}
\end{Verbatim}
\end{tcolorbox}

    We show a preview of the data sets. Firstly, we show the head of
\texttt{pueblos\_magicos.csv}.

    \begin{tcolorbox}[breakable, size=fbox, boxrule=1pt, pad at break*=1mm,colback=cellbackground, colframe=cellborder]
\prompt{In}{incolor}{2}{\boxspacing}
\begin{Verbatim}[commandchars=\\\{\}]
\PY{n}{magic\PYZus{}towns}\PY{o}{.}\PY{n}{head}\PY{p}{(}\PY{p}{)}
\end{Verbatim}
\end{tcolorbox}

            \begin{tcolorbox}[breakable, size=fbox, boxrule=.5pt, pad at break*=1mm, opacityfill=0]
\prompt{Out}{outcolor}{2}{\boxspacing}
\begin{Verbatim}[commandchars=\\\{\}]
        pueblo\_magico               estado  pob\_2010  pob\_2015
0            Asientos       Aguascalientes     48358     50864
1            Calvillo       Aguascalientes     57627     60760
2  San José de Gracia       Aguascalientes      7160      9661
3              Tecate      Baja California     89999    110870
4              La Paz  Baja California Sur    265717    293687
\end{Verbatim}
\end{tcolorbox}
        
    Then, we show the head of \texttt{turismo\_mexico.csv} dataset.

    \begin{tcolorbox}[breakable, size=fbox, boxrule=1pt, pad at break*=1mm,colback=cellbackground, colframe=cellborder]
\prompt{In}{incolor}{3}{\boxspacing}
\begin{Verbatim}[commandchars=\\\{\}]
\PY{n}{mexico\PYZus{}tourism}\PY{o}{.}\PY{n}{head}\PY{p}{(}\PY{p}{)}
\end{Verbatim}
\end{tcolorbox}

            \begin{tcolorbox}[breakable, size=fbox, boxrule=.5pt, pad at break*=1mm, opacityfill=0]
\prompt{Out}{outcolor}{3}{\boxspacing}
\begin{Verbatim}[commandchars=\\\{\}]
      fecha  visitantes\_internacionales  turismo\_al\_interior  \textbackslash{}
0  01/01/16                        7808                 1690
1  01/02/16                        7666                 1683
2  01/03/16                        8625                 1983
3  01/04/16                        7717                 1601
4  01/05/16                        7665                 1548

   turismo\_fronterizo  excursionistas \_fronterizos  pasajeros\_crucero
0                1152                         4332                634
1                1048                         4250                685
2                1224                         4678                739
3                1083                         4451                582
4                1154                         4538                424
\end{Verbatim}
\end{tcolorbox}
        
    \hypertarget{exploratory-data-analysis}{%
\subsubsection{Exploratory Data
Analysis}\label{exploratory-data-analysis}}

    In the following we are going to answer a pair of questions using the
datasets.

    \hypertarget{what-were-the-ten-pueblos-muxe1gicos-with-the-most-population-in-2015}{%
\paragraph{What were the ten Pueblos mágicos with the most population in
2015?}\label{what-were-the-ten-pueblos-muxe1gicos-with-the-most-population-in-2015}}

    In order to answer this question we sort the dataset by that column
storing that information in a new variable, and then we obtain the firts
then rows corresponding to ten Pueblos máxicos with de most population
in 2015.

    \begin{tcolorbox}[breakable, size=fbox, boxrule=1pt, pad at break*=1mm,colback=cellbackground, colframe=cellborder]
\prompt{In}{incolor}{4}{\boxspacing}
\begin{Verbatim}[commandchars=\\\{\}]
\PY{n}{pob\PYZus{}2015\PYZus{}magic\PYZus{}towns}\PY{o}{=}\PY{n}{magic\PYZus{}towns}\PY{o}{.}\PY{n}{sort\PYZus{}values}\PY{p}{(}\PY{p}{[}\PY{l+s+s1}{\PYZsq{}}\PY{l+s+s1}{pob\PYZus{}2015}\PY{l+s+s1}{\PYZsq{}}\PY{p}{]}\PY{p}{,}\PY{n}{ascending}\PY{o}{=}\PY{k+kc}{False}\PY{p}{)}
\PY{n}{pob\PYZus{}10\PYZus{}towns\PYZus{}most}\PY{o}{=}\PY{n}{pob\PYZus{}2015\PYZus{}magic\PYZus{}towns}\PY{p}{[}\PY{l+s+s1}{\PYZsq{}}\PY{l+s+s1}{pueblo\PYZus{}magico}\PY{l+s+s1}{\PYZsq{}}\PY{p}{]}\PY{p}{[}\PY{p}{:}\PY{l+m+mi}{10}\PY{p}{]}
\PY{n+nb}{print}\PY{p}{(}\PY{n}{pob\PYZus{}10\PYZus{}towns\PYZus{}most}\PY{p}{)}
\end{Verbatim}
\end{tcolorbox}

    \begin{Verbatim}[commandchars=\\\{\}]
44                      Tlaquepaque
4                            La Paz
48                          Metepec
18       San Cristóbal de las Casas
117                       Guadalupe
113                        Papantla
15             Comitán de Domínguez
37                  Lagos de Moreno
81               San Andrés Cholula
67     Bahía de Banderas (Sayulita)
Name: pueblo\_magico, dtype: object
    \end{Verbatim}

    The last list shows the 10 Pueblos mágicos with the most population in
2015.

    \hypertarget{what-were-the-ten-pueblos-muxe1gicos-with-de-least-population-in-2010}{%
\paragraph{What were the ten Pueblos mágicos with de least population in
2010?}\label{what-were-the-ten-pueblos-muxe1gicos-with-de-least-population-in-2010}}

    We do the same, store in a variable the dataframe of Pueblos mágicos
ordered in base of \texttt{pob\_2010} column, then we print the
registers of interest.

    \begin{tcolorbox}[breakable, size=fbox, boxrule=1pt, pad at break*=1mm,colback=cellbackground, colframe=cellborder]
\prompt{In}{incolor}{5}{\boxspacing}
\begin{Verbatim}[commandchars=\\\{\}]
\PY{n}{pob\PYZus{}2010\PYZus{}magic\PYZus{}towns}\PY{o}{=}\PY{n}{magic\PYZus{}towns}\PY{o}{.}\PY{n}{sort\PYZus{}values}\PY{p}{(}\PY{p}{[}\PY{l+s+s1}{\PYZsq{}}\PY{l+s+s1}{pob\PYZus{}2010}\PY{l+s+s1}{\PYZsq{}}\PY{p}{]}\PY{p}{)}
\PY{n}{pob\PYZus{}10\PYZus{}towns\PYZus{}least}\PY{o}{=}\PY{n}{pob\PYZus{}2010\PYZus{}magic\PYZus{}towns}\PY{p}{[}\PY{l+s+s1}{\PYZsq{}}\PY{l+s+s1}{pueblo\PYZus{}magico}\PY{l+s+s1}{\PYZsq{}}\PY{p}{]}\PY{p}{[}\PY{p}{:}\PY{l+m+mi}{10}\PY{p}{]}
\PY{n+nb}{print}\PY{p}{(}\PY{n}{pob\PYZus{}10\PYZus{}towns\PYZus{}least}\PY{p}{)}
\end{Verbatim}
\end{tcolorbox}

    \begin{Verbatim}[commandchars=\\\{\}]
72                  Capulálpam de Méndez
8                                Candela
10                              Guerrero
68                            Bustamante
105                                 Mier
74     San Pedro y San Pablo Teposcolula
122              Teúl de González Ortega
40               San Sebastián del Oeste
2                     San José de Gracia
33                     Mineral del Chico
Name: pueblo\_magico, dtype: object
    \end{Verbatim}

    These are the ten Pueblos mágicos with the least population in 2010.

    \hypertarget{data-wrangling}{%
\subsubsection{Data Wrangling}\label{data-wrangling}}

    We create a \texttt{DataFrame} with the table in the exam's PDF
containing the Full name and the corresponding ISO code.

    \begin{tcolorbox}[breakable, size=fbox, boxrule=1pt, pad at break*=1mm,colback=cellbackground, colframe=cellborder]
\prompt{In}{incolor}{6}{\boxspacing}
\begin{Verbatim}[commandchars=\\\{\}]
\PY{n}{df\PYZus{}estado\PYZus{}iso}\PY{o}{=}\PY{n}{pd}\PY{o}{.}\PY{n}{read\PYZus{}csv}\PY{p}{(}\PY{l+s+s1}{\PYZsq{}}\PY{l+s+s1}{estado\PYZus{}iso.csv}\PY{l+s+s1}{\PYZsq{}}\PY{p}{)}
\PY{n}{df\PYZus{}estado\PYZus{}iso}\PY{o}{.}\PY{n}{sort\PYZus{}values}\PY{p}{(}\PY{p}{[}\PY{l+s+s1}{\PYZsq{}}\PY{l+s+s1}{Full name}\PY{l+s+s1}{\PYZsq{}}\PY{p}{]}\PY{p}{,}\PY{n}{inplace}\PY{o}{=}\PY{k+kc}{True}\PY{p}{)}
\PY{n}{df\PYZus{}estado\PYZus{}iso}\PY{o}{.}\PY{n}{head}\PY{p}{(}\PY{p}{)}
\end{Verbatim}
\end{tcolorbox}

            \begin{tcolorbox}[breakable, size=fbox, boxrule=.5pt, pad at break*=1mm, opacityfill=0]
\prompt{Out}{outcolor}{6}{\boxspacing}
\begin{Verbatim}[commandchars=\\\{\}]
             Full name 3-letter-code
0       Aguascalientes           AGU
1      Baja California           BCN
2  Baja California Sur           BCS
3             Campeche           CAM
6              Chiapas           CHP
\end{Verbatim}
\end{tcolorbox}
        
    First of all, we are going to check if the data in \texttt{Full\ name}
is equal to the set of states storing in \texttt{estado} column in the
original \texttt{pueblos\_magicos.csv}. In the next cell, we charge the
different registers of \texttt{df\_estado\_iso} and
\texttt{pueblos\_magicos.csv}and check if the two list of states are
equal

    \begin{tcolorbox}[breakable, size=fbox, boxrule=1pt, pad at break*=1mm,colback=cellbackground, colframe=cellborder]
\prompt{In}{incolor}{7}{\boxspacing}
\begin{Verbatim}[commandchars=\\\{\}]
\PY{n}{states\PYZus{}full\PYZus{}name}\PY{o}{=}\PY{n}{df\PYZus{}estado\PYZus{}iso}\PY{p}{[}\PY{l+s+s1}{\PYZsq{}}\PY{l+s+s1}{Full name}\PY{l+s+s1}{\PYZsq{}}\PY{p}{]}\PY{o}{.}\PY{n}{to\PYZus{}list}\PY{p}{(}\PY{p}{)}
\PY{n}{states\PYZus{}full\PYZus{}name}\PY{o}{.}\PY{n}{sort}\PY{p}{(}\PY{p}{)}
\PY{n}{states\PYZus{}original}\PY{o}{=}\PY{n+nb}{list}\PY{p}{(}\PY{n+nb}{set}\PY{p}{(}\PY{n}{magic\PYZus{}towns}\PY{p}{[}\PY{l+s+s1}{\PYZsq{}}\PY{l+s+s1}{estado}\PY{l+s+s1}{\PYZsq{}}\PY{p}{]}\PY{p}{)}\PY{p}{)}
\PY{n}{states\PYZus{}original}\PY{o}{.}\PY{n}{sort}\PY{p}{(}\PY{p}{)}
\PY{n+nb}{print}\PY{p}{(}\PY{n}{states\PYZus{}full\PYZus{}name}\PY{o}{==}\PY{n}{states\PYZus{}original}\PY{p}{)}
\end{Verbatim}
\end{tcolorbox}

    \begin{Verbatim}[commandchars=\\\{\}]
False
    \end{Verbatim}

    The last result is \texttt{False}, then we are going to check the set of
states in the original data set \texttt{pueblos\_magicos.csv}.

    \begin{tcolorbox}[breakable, size=fbox, boxrule=1pt, pad at break*=1mm,colback=cellbackground, colframe=cellborder]
\prompt{In}{incolor}{8}{\boxspacing}
\begin{Verbatim}[commandchars=\\\{\}]
\PY{n}{states\PYZus{}original}
\end{Verbatim}
\end{tcolorbox}

            \begin{tcolorbox}[breakable, size=fbox, boxrule=.5pt, pad at break*=1mm, opacityfill=0]
\prompt{Out}{outcolor}{8}{\boxspacing}
\begin{Verbatim}[commandchars=\\\{\}]
['Aguascalientes',
 'Baja California',
 'Baja California Sur',
 'Campeche',
 'Chiapas',
 'Chihuahua',
 'Coahuila',
 'Colima',
 'Durango',
 'Guanajuato',
 'Guerrero',
 'Hidalgo',
 'Jalisco',
 'Mexico',
 'Michoacan',
 'Morelos',
 'Nayarit',
 'Nuevo Leon',
 'Oaxaca',
 'Puebla',
 'Queretaro',
 'Quintana Roo',
 'San Luis Potosi',
 'Sinaloa',
 'Sonora',
 'Tabasco',
 'Tamaulipas',
 'Tlaxcala',
 'Veracruz',
 'Yucatan',
 'Zacatecas',
 'Zacatezas']
\end{Verbatim}
\end{tcolorbox}
        
    We can see a problem with a typo in Zacatecas spelling, therefore exists
a row with a state \texttt{Zacatezas} instead of \texttt{Zacatecas}. We
are going to repair that mistake in the dataset. We will replace
\texttt{Zacatezas} with \texttt{Zacatecas} and then check if the set of
states of the setting dataset is equal to the list of states in the
table \texttt{df\_estado\_iso}.

    \begin{tcolorbox}[breakable, size=fbox, boxrule=1pt, pad at break*=1mm,colback=cellbackground, colframe=cellborder]
\prompt{In}{incolor}{9}{\boxspacing}
\begin{Verbatim}[commandchars=\\\{\}]
\PY{n}{magic\PYZus{}towns}\PY{o}{.}\PY{n}{loc}\PY{p}{[}\PY{n}{magic\PYZus{}towns}\PY{p}{[}\PY{l+s+s1}{\PYZsq{}}\PY{l+s+s1}{estado}\PY{l+s+s1}{\PYZsq{}}\PY{p}{]}\PY{o}{==}\PY{l+s+s1}{\PYZsq{}}\PY{l+s+s1}{Zacatezas}\PY{l+s+s1}{\PYZsq{}}\PY{p}{,}\PY{l+s+s1}{\PYZsq{}}\PY{l+s+s1}{estado}\PY{l+s+s1}{\PYZsq{}}\PY{p}{]}\PY{o}{=}\PY{l+s+s1}{\PYZsq{}}\PY{l+s+s1}{Zacatecas}\PY{l+s+s1}{\PYZsq{}}
\PY{n}{states\PYZus{}original\PYZus{}correct}\PY{o}{=}\PY{n+nb}{list}\PY{p}{(}\PY{n+nb}{set}\PY{p}{(}\PY{n}{magic\PYZus{}towns}\PY{p}{[}\PY{l+s+s1}{\PYZsq{}}\PY{l+s+s1}{estado}\PY{l+s+s1}{\PYZsq{}}\PY{p}{]}\PY{p}{)}\PY{p}{)}
\PY{n}{states\PYZus{}original\PYZus{}correct}\PY{o}{.}\PY{n}{sort}\PY{p}{(}\PY{p}{)}
\PY{n+nb}{print}\PY{p}{(}\PY{n}{states\PYZus{}original\PYZus{}correct}\PY{o}{==}\PY{n}{states\PYZus{}full\PYZus{}name}\PY{p}{)}
\end{Verbatim}
\end{tcolorbox}

    \begin{Verbatim}[commandchars=\\\{\}]
True
    \end{Verbatim}

    Finally, with the same set of states in the original dataset and
\texttt{df\_estado\_iso} we can set the \texttt{estado} column with the
corresponding ISO code using a dictionary with de \texttt{Full\ name} as
a key and \texttt{3-letter-code} as a value.

    \begin{tcolorbox}[breakable, size=fbox, boxrule=1pt, pad at break*=1mm,colback=cellbackground, colframe=cellborder]
\prompt{In}{incolor}{10}{\boxspacing}
\begin{Verbatim}[commandchars=\\\{\}]
\PY{n}{dic\PYZus{}states\PYZus{}original\PYZus{}correct}\PY{o}{=}\PY{n}{pd}\PY{o}{.}\PY{n}{Series}\PY{p}{(}\PY{n}{df\PYZus{}estado\PYZus{}iso}\PY{p}{[}\PY{l+s+s1}{\PYZsq{}}\PY{l+s+s1}{3\PYZhy{}letter\PYZhy{}code}\PY{l+s+s1}{\PYZsq{}}\PY{p}{]}\PY{o}{.}\PY{n}{values}\PY{p}{,}\PY{n}{index}\PY{o}{=}\PY{n}{df\PYZus{}estado\PYZus{}iso}\PY{p}{[}\PY{l+s+s1}{\PYZsq{}}\PY{l+s+s1}{Full name}\PY{l+s+s1}{\PYZsq{}}\PY{p}{]}\PY{p}{)}\PY{o}{.}\PY{n}{to\PYZus{}dict}\PY{p}{(}\PY{p}{)}
\PY{k}{for} \PY{n}{state} \PY{o+ow}{in} \PY{n}{states\PYZus{}full\PYZus{}name}\PY{p}{:}
    \PY{n}{magic\PYZus{}towns}\PY{o}{.}\PY{n}{loc}\PY{p}{[}\PY{n}{magic\PYZus{}towns}\PY{p}{[}\PY{l+s+s1}{\PYZsq{}}\PY{l+s+s1}{estado}\PY{l+s+s1}{\PYZsq{}}\PY{p}{]}\PY{o}{==}\PY{n}{state}\PY{p}{,}\PY{l+s+s1}{\PYZsq{}}\PY{l+s+s1}{estado}\PY{l+s+s1}{\PYZsq{}}\PY{p}{]}\PY{o}{=}\PY{n}{dic\PYZus{}states\PYZus{}original\PYZus{}correct}\PY{p}{[}\PY{n}{state}\PY{p}{]}
\PY{n}{magic\PYZus{}towns}\PY{o}{.}\PY{n}{head}\PY{p}{(}\PY{p}{)}
\end{Verbatim}
\end{tcolorbox}

            \begin{tcolorbox}[breakable, size=fbox, boxrule=.5pt, pad at break*=1mm, opacityfill=0]
\prompt{Out}{outcolor}{10}{\boxspacing}
\begin{Verbatim}[commandchars=\\\{\}]
        pueblo\_magico estado  pob\_2010  pob\_2015
0            Asientos    AGU     48358     50864
1            Calvillo    AGU     57627     60760
2  San José de Gracia    AGU      7160      9661
3              Tecate    BCN     89999    110870
4              La Paz    BCS    265717    293687
\end{Verbatim}
\end{tcolorbox}
        
    \hypertarget{analysis}{%
\subsubsection{Analysis}\label{analysis}}

    In this section, we show a executive summary for the historical
evolution of International tourism in Mexico.

    Before to begin, we see the main info and check the differents
cathegories of tourism.

    \begin{tcolorbox}[breakable, size=fbox, boxrule=1pt, pad at break*=1mm,colback=cellbackground, colframe=cellborder]
\prompt{In}{incolor}{11}{\boxspacing}
\begin{Verbatim}[commandchars=\\\{\}]
\PY{n}{mexico\PYZus{}tourism}\PY{p}{[}\PY{l+s+s1}{\PYZsq{}}\PY{l+s+s1}{fecha}\PY{l+s+s1}{\PYZsq{}}\PY{p}{]} \PY{o}{=} \PY{n}{pd}\PY{o}{.}\PY{n}{to\PYZus{}datetime}\PY{p}{(}\PY{n}{mexico\PYZus{}tourism}\PY{p}{[}\PY{l+s+s1}{\PYZsq{}}\PY{l+s+s1}{fecha}\PY{l+s+s1}{\PYZsq{}}\PY{p}{]}\PY{p}{,}\PY{n}{dayfirst}\PY{o}{=}\PY{k+kc}{True}\PY{p}{)}
\PY{n}{mexico\PYZus{}tourism}\PY{o}{.}\PY{n}{info}\PY{p}{(}\PY{p}{)}
\end{Verbatim}
\end{tcolorbox}

    \begin{Verbatim}[commandchars=\\\{\}]
<class 'pandas.core.frame.DataFrame'>
RangeIndex: 71 entries, 0 to 70
Data columns (total 6 columns):
 \#   Column                       Non-Null Count  Dtype
---  ------                       --------------  -----
 0   fecha                        71 non-null     datetime64[ns]
 1   visitantes\_internacionales   71 non-null     int64
 2   turismo\_al\_interior          71 non-null     int64
 3   turismo\_fronterizo           71 non-null     int64
 4   excursionistas \_fronterizos  71 non-null     int64
 5   pasajeros\_crucero            71 non-null     int64
dtypes: datetime64[ns](1), int64(5)
memory usage: 3.5 KB
    \end{Verbatim}

    We have four tourism cathegories:

\begin{enumerate}
\def\labelenumi{\arabic{enumi}.}
\tightlist
\item
  turismo al interior
\item
  turismo fronterizo
\item
  excursionistas fronterizos
\item
  pasajeros crucero
\end{enumerate}

From the dataset we obtain the information of what years and months we
have information of international tourism.

    \begin{tcolorbox}[breakable, size=fbox, boxrule=1pt, pad at break*=1mm,colback=cellbackground, colframe=cellborder]
\prompt{In}{incolor}{12}{\boxspacing}
\begin{Verbatim}[commandchars=\\\{\}]
\PY{n}{year\PYZus{}tourism}\PY{o}{=}\PY{n}{mexico\PYZus{}tourism}\PY{p}{[}\PY{l+s+s1}{\PYZsq{}}\PY{l+s+s1}{fecha}\PY{l+s+s1}{\PYZsq{}}\PY{p}{]}\PY{o}{.}\PY{n}{dt}\PY{o}{.}\PY{n}{year}
\PY{n}{month\PYZus{}tourism}\PY{o}{=}\PY{n}{mexico\PYZus{}tourism}\PY{p}{[}\PY{l+s+s1}{\PYZsq{}}\PY{l+s+s1}{fecha}\PY{l+s+s1}{\PYZsq{}}\PY{p}{]}\PY{o}{.}\PY{n}{dt}\PY{o}{.}\PY{n}{month}
\PY{n}{mexico\PYZus{}tourism}\PY{p}{[}\PY{l+s+s1}{\PYZsq{}}\PY{l+s+s1}{year}\PY{l+s+s1}{\PYZsq{}}\PY{p}{]}\PY{o}{=}\PY{n}{year\PYZus{}tourism}
\PY{n}{mexico\PYZus{}tourism}\PY{p}{[}\PY{l+s+s1}{\PYZsq{}}\PY{l+s+s1}{month}\PY{l+s+s1}{\PYZsq{}}\PY{p}{]}\PY{o}{=}\PY{n}{month\PYZus{}tourism}
\PY{n}{set\PYZus{}years\PYZus{}tourism}\PY{o}{=}\PY{n+nb}{set}\PY{p}{(}\PY{n}{year\PYZus{}tourism}\PY{p}{)}
\PY{n}{set\PYZus{}months\PYZus{}tourism}\PY{o}{=}\PY{n+nb}{set}\PY{p}{(}\PY{n}{month\PYZus{}tourism}\PY{p}{)}
\PY{n+nb}{print}\PY{p}{(}\PY{n}{set\PYZus{}years\PYZus{}tourism}\PY{p}{)}
\PY{n+nb}{print}\PY{p}{(}\PY{n}{set\PYZus{}months\PYZus{}tourism}\PY{p}{)}
\end{Verbatim}
\end{tcolorbox}

    \begin{Verbatim}[commandchars=\\\{\}]
\{2016, 2017, 2018, 2019, 2020, 2021\}
\{1, 2, 3, 4, 5, 6, 7, 8, 9, 10, 11, 12\}
    \end{Verbatim}

    \begin{tcolorbox}[breakable, size=fbox, boxrule=1pt, pad at break*=1mm,colback=cellbackground, colframe=cellborder]
\prompt{In}{incolor}{13}{\boxspacing}
\begin{Verbatim}[commandchars=\\\{\}]
\PY{n+nb}{print}\PY{p}{(}\PY{l+s+s1}{\PYZsq{}}\PY{l+s+s1}{First day of registration }\PY{l+s+s1}{\PYZsq{}}\PY{p}{,}\PY{n}{mexico\PYZus{}tourism}\PY{o}{.}\PY{n}{iloc}\PY{p}{[}\PY{l+m+mi}{0}\PY{p}{]}\PY{p}{[}\PY{l+s+s1}{\PYZsq{}}\PY{l+s+s1}{fecha}\PY{l+s+s1}{\PYZsq{}}\PY{p}{]}\PY{p}{)}
\PY{n+nb}{print}\PY{p}{(}\PY{l+s+s1}{\PYZsq{}}\PY{l+s+s1}{Last day of registration }\PY{l+s+s1}{\PYZsq{}}\PY{p}{,}\PY{n}{mexico\PYZus{}tourism}\PY{o}{.}\PY{n}{iloc}\PY{p}{[}\PY{o}{\PYZhy{}}\PY{l+m+mi}{1}\PY{p}{]}\PY{p}{[}\PY{l+s+s1}{\PYZsq{}}\PY{l+s+s1}{fecha}\PY{l+s+s1}{\PYZsq{}}\PY{p}{]}\PY{p}{)}
\end{Verbatim}
\end{tcolorbox}

    \begin{Verbatim}[commandchars=\\\{\}]
First day of registration  2016-01-01 00:00:00
Last day of registration  2021-11-01 00:00:00
    \end{Verbatim}

    First, we graph the evolution of total tourism from \texttt{01/01/2016}
to \texttt{01/11/2021}

    \begin{tcolorbox}[breakable, size=fbox, boxrule=1pt, pad at break*=1mm,colback=cellbackground, colframe=cellborder]
\prompt{In}{incolor}{14}{\boxspacing}
\begin{Verbatim}[commandchars=\\\{\}]
\PY{k+kn}{from} \PY{n+nn}{plotnine} \PY{k+kn}{import} \PY{o}{*}
\PY{n}{gg1}\PY{o}{=}\PY{n}{ggplot}\PY{p}{(}\PY{n}{mexico\PYZus{}tourism}\PY{p}{)}\PY{o}{+}\PY{n}{geom\PYZus{}line}\PY{p}{(}\PY{n}{aes}\PY{p}{(}\PY{n}{x}\PY{o}{=}\PY{l+s+s1}{\PYZsq{}}\PY{l+s+s1}{fecha}\PY{l+s+s1}{\PYZsq{}}\PY{p}{,}\PY{n}{y}\PY{o}{=}\PY{l+s+s1}{\PYZsq{}}\PY{l+s+s1}{visitantes\PYZus{}internacionales}\PY{l+s+s1}{\PYZsq{}}\PY{p}{)}\PY{p}{)}\PY{o}{+}\PY{n}{theme}\PY{p}{(}\PY{n}{figure\PYZus{}size}\PY{o}{=}\PY{p}{(}\PY{l+m+mi}{8}\PY{p}{,}\PY{l+m+mi}{5}\PY{p}{)}\PY{p}{)}\PY{o}{+}\PY{n}{labs}\PY{p}{(}\PY{n}{x}\PY{o}{=}\PY{l+s+s1}{\PYZsq{}}\PY{l+s+s1}{year}\PY{l+s+s1}{\PYZsq{}}\PY{p}{,}\PY{n}{y}\PY{o}{=}\PY{l+s+s1}{\PYZsq{}}\PY{l+s+s1}{tourists (x1000)}\PY{l+s+s1}{\PYZsq{}}\PY{p}{)}\PY{o}{+}\PY{n}{scale\PYZus{}x\PYZus{}date}\PY{p}{(}\PY{n}{date\PYZus{}breaks}\PY{o}{=}\PY{l+s+s1}{\PYZsq{}}\PY{l+s+s1}{1 year}\PY{l+s+s1}{\PYZsq{}}\PY{p}{,}\PY{n}{date\PYZus{}labels}\PY{o}{=}\PY{l+s+s1}{\PYZsq{}}\PY{l+s+s1}{\PYZpc{}}\PY{l+s+s1}{y}\PY{l+s+s1}{\PYZsq{}}\PY{p}{)}
\PY{n}{gg1}
\end{Verbatim}
\end{tcolorbox}

    \begin{center}
    \adjustimage{max size={0.9\linewidth}{0.9\paperheight}}{Take-Home-test-Nu-Roberto_files/Take-Home-test-Nu-Roberto_37_0.png}
    \end{center}
    { \hspace*{\fill} \\}
    
            \begin{tcolorbox}[breakable, size=fbox, boxrule=.5pt, pad at break*=1mm, opacityfill=0]
\prompt{Out}{outcolor}{14}{\boxspacing}
\begin{Verbatim}[commandchars=\\\{\}]
<ggplot: (8754084237210)>
\end{Verbatim}
\end{tcolorbox}
        
    Afterwards, we graph the evolution of the international tourism per
cathegory in all the time range we have.

    \begin{tcolorbox}[breakable, size=fbox, boxrule=1pt, pad at break*=1mm,colback=cellbackground, colframe=cellborder]
\prompt{In}{incolor}{15}{\boxspacing}
\begin{Verbatim}[commandchars=\\\{\}]
\PY{n}{colors\PYZus{}palette}\PY{o}{=}\PY{p}{[}\PY{l+s+s2}{\PYZdq{}}\PY{l+s+s2}{\PYZsh{}91A3E1}\PY{l+s+s2}{\PYZdq{}}\PY{p}{,} \PY{l+s+s2}{\PYZdq{}}\PY{l+s+s2}{\PYZsh{}83B692}\PY{l+s+s2}{\PYZdq{}}\PY{p}{,}\PY{l+s+s2}{\PYZdq{}}\PY{l+s+s2}{\PYZsh{}F9ADA0}\PY{l+s+s2}{\PYZdq{}}\PY{p}{,} \PY{l+s+s2}{\PYZdq{}}\PY{l+s+s2}{\PYZsh{}CD6FD5}\PY{l+s+s2}{\PYZdq{}}\PY{p}{]}
\PY{n}{tourism\PYZus{}cathegory}\PY{o}{=}\PY{n+nb}{list}\PY{p}{(}\PY{n}{mexico\PYZus{}tourism}\PY{o}{.}\PY{n}{columns}\PY{p}{[}\PY{l+m+mi}{2}\PY{p}{:}\PY{l+m+mi}{6}\PY{p}{]}\PY{p}{)}
\PY{n}{dfm\PYZus{}mexico\PYZus{}tourism} \PY{o}{=} \PY{n}{mexico\PYZus{}tourism}\PY{o}{.}\PY{n}{melt}\PY{p}{(}\PY{n}{id\PYZus{}vars}\PY{o}{=}\PY{p}{[}\PY{l+s+s1}{\PYZsq{}}\PY{l+s+s1}{year}\PY{l+s+s1}{\PYZsq{}}\PY{p}{,} \PY{l+s+s1}{\PYZsq{}}\PY{l+s+s1}{fecha}\PY{l+s+s1}{\PYZsq{}}\PY{p}{]}\PY{p}{,} \PY{n}{value\PYZus{}vars}\PY{o}{=}\PY{n}{tourism\PYZus{}cathegory}\PY{p}{)}
\PY{n}{gg2}\PY{o}{=}\PY{n}{ggplot}\PY{p}{(}\PY{n}{dfm\PYZus{}mexico\PYZus{}tourism}\PY{p}{)}\PY{o}{+}\PY{n}{geom\PYZus{}line}\PY{p}{(}\PY{n}{aes}\PY{p}{(}\PY{n}{x}\PY{o}{=}\PY{l+s+s1}{\PYZsq{}}\PY{l+s+s1}{fecha}\PY{l+s+s1}{\PYZsq{}}\PY{p}{,}\PY{n}{y}\PY{o}{=}\PY{l+s+s1}{\PYZsq{}}\PY{l+s+s1}{value}\PY{l+s+s1}{\PYZsq{}}\PY{p}{,}\PY{n}{color}\PY{o}{=}\PY{l+s+s1}{\PYZsq{}}\PY{l+s+s1}{variable}\PY{l+s+s1}{\PYZsq{}}\PY{p}{,}\PY{n}{group}\PY{o}{=}\PY{l+s+s1}{\PYZsq{}}\PY{l+s+s1}{variable}\PY{l+s+s1}{\PYZsq{}}\PY{p}{)}\PY{p}{,}\PY{n}{size}\PY{o}{=}\PY{l+m+mi}{1}\PY{p}{)}\PY{o}{+}\PY{n}{theme}\PY{p}{(}\PY{n}{figure\PYZus{}size}\PY{o}{=}\PY{p}{(}\PY{l+m+mi}{10}\PY{p}{,}\PY{l+m+mi}{5}\PY{p}{)}\PY{p}{)}\PY{o}{+}\PY{n}{scale\PYZus{}x\PYZus{}date}\PY{p}{(}\PY{n}{date\PYZus{}breaks}\PY{o}{=}\PY{l+s+s1}{\PYZsq{}}\PY{l+s+s1}{1 year}\PY{l+s+s1}{\PYZsq{}}\PY{p}{,}\PY{n}{date\PYZus{}labels}\PY{o}{=}\PY{l+s+s1}{\PYZsq{}}\PY{l+s+s1}{\PYZpc{}}\PY{l+s+s1}{y}\PY{l+s+s1}{\PYZsq{}}\PY{p}{)}\PY{o}{+}\PY{n}{theme}\PY{p}{(}\PY{n}{figure\PYZus{}size}\PY{o}{=}\PY{p}{(}\PY{l+m+mi}{8}\PY{p}{,}\PY{l+m+mi}{5}\PY{p}{)}\PY{p}{)}\PY{o}{+}\PY{n}{labs}\PY{p}{(}\PY{n}{x}\PY{o}{=}\PY{l+s+s1}{\PYZsq{}}\PY{l+s+s1}{date}\PY{l+s+s1}{\PYZsq{}}\PY{p}{,} \PY{n}{y}\PY{o}{=}\PY{l+s+s1}{\PYZsq{}}\PY{l+s+s1}{tourists (x1000)}\PY{l+s+s1}{\PYZsq{}}\PY{p}{)}\PY{o}{+}\PY{n}{scale\PYZus{}color\PYZus{}manual}\PY{p}{(}\PY{n}{values}\PY{o}{=}\PY{n}{colors\PYZus{}palette}\PY{p}{,}\PY{n}{name}\PY{o}{=}\PY{l+s+s1}{\PYZsq{}}\PY{l+s+s1}{tourism type}\PY{l+s+s1}{\PYZsq{}}\PY{p}{,}\PY{n}{labels}\PY{o}{=}\PY{p}{[}\PY{l+s+s1}{\PYZsq{}}\PY{l+s+s1}{EF}\PY{l+s+s1}{\PYZsq{}}\PY{p}{,}\PY{l+s+s1}{\PYZsq{}}\PY{l+s+s1}{PC}\PY{l+s+s1}{\PYZsq{}}\PY{p}{,}\PY{l+s+s1}{\PYZsq{}}\PY{l+s+s1}{TI}\PY{l+s+s1}{\PYZsq{}}\PY{p}{,}\PY{l+s+s1}{\PYZsq{}}\PY{l+s+s1}{TF}\PY{l+s+s1}{\PYZsq{}}\PY{p}{]}\PY{p}{)}
\PY{n}{gg2}
\end{Verbatim}
\end{tcolorbox}

    \begin{center}
    \adjustimage{max size={0.9\linewidth}{0.9\paperheight}}{Take-Home-test-Nu-Roberto_files/Take-Home-test-Nu-Roberto_39_0.png}
    \end{center}
    { \hspace*{\fill} \\}
    
            \begin{tcolorbox}[breakable, size=fbox, boxrule=.5pt, pad at break*=1mm, opacityfill=0]
\prompt{Out}{outcolor}{15}{\boxspacing}
\begin{Verbatim}[commandchars=\\\{\}]
<ggplot: (8754084144628)>
\end{Verbatim}
\end{tcolorbox}
        
    Where EF,PC,TI, and TF correspond to \emph{excursionistas
fronterizos},\emph{pasajeros en crucero},\emph{turismo al interior}, and
\emph{turismo fronterizo} respectively.

We see the evolution of the tourism from 2016 to 2021. Clearly, at the
beginning of 2020 we observe an abrupt decreasing of the four tourism
cathegories. It could be for the COVID-19 pandemic. Moreover, we observe
a in different proportion each cathegory after 2020. Before 2020 we
observed in decreasing order of proportion EF, TI, TF, and PC.

    We continue to evaluate the distribution of each tourism cathegory. We
are going to see this with a boxplot.

    \begin{tcolorbox}[breakable, size=fbox, boxrule=1pt, pad at break*=1mm,colback=cellbackground, colframe=cellborder]
\prompt{In}{incolor}{16}{\boxspacing}
\begin{Verbatim}[commandchars=\\\{\}]
\PY{k+kn}{import} \PY{n+nn}{numpy} \PY{k}{as} \PY{n+nn}{np}
\PY{k}{def} \PY{n+nf}{mean}\PY{p}{(}\PY{n}{x}\PY{p}{)}\PY{p}{:}
    \PY{k}{return} \PY{n}{np}\PY{o}{.}\PY{n}{mean}\PY{p}{(}\PY{n}{x}\PY{p}{)}
\PY{n}{means}\PY{o}{=}\PY{n}{pd}\PY{o}{.}\PY{n}{DataFrame}\PY{p}{(}\PY{n}{mexico\PYZus{}tourism}\PY{p}{[}\PY{n}{tourism\PYZus{}cathegory}\PY{p}{]}\PY{o}{.}\PY{n}{mean}\PY{p}{(}\PY{p}{)}\PY{p}{)}
\PY{c+c1}{\PYZsh{}mexico\PYZus{}tourism.boxplot(column=tourism\PYZus{}cathegory,figsize=(10,7),showmeans=True)}
\PY{n}{gg3}\PY{o}{=}\PY{n}{ggplot}\PY{p}{(}\PY{n}{dfm\PYZus{}mexico\PYZus{}tourism}\PY{p}{)}\PY{o}{+}\PY{n}{geom\PYZus{}boxplot}\PY{p}{(}\PY{n}{aes}\PY{p}{(}\PY{l+s+s1}{\PYZsq{}}\PY{l+s+s1}{variable}\PY{l+s+s1}{\PYZsq{}}\PY{p}{,}\PY{l+s+s1}{\PYZsq{}}\PY{l+s+s1}{value}\PY{l+s+s1}{\PYZsq{}}\PY{p}{)}\PY{p}{)}\PY{o}{+}\PY{n}{labs}\PY{p}{(}\PY{n}{x}\PY{o}{=}\PY{l+s+s1}{\PYZsq{}}\PY{l+s+s1}{tourism type}\PY{l+s+s1}{\PYZsq{}}\PY{p}{,}\PY{n}{y}\PY{o}{=}\PY{l+s+s1}{\PYZsq{}}\PY{l+s+s1}{tourists (x1000)}\PY{l+s+s1}{\PYZsq{}}\PY{p}{)}
\PY{n}{gg3}\PY{o}{=} \PY{p}{(}
    \PY{n}{ggplot}\PY{p}{(}\PY{n}{dfm\PYZus{}mexico\PYZus{}tourism}\PY{p}{,} \PY{n}{aes}\PY{p}{(}\PY{n}{x}\PY{o}{=}\PY{l+s+s2}{\PYZdq{}}\PY{l+s+s2}{variable}\PY{l+s+s2}{\PYZdq{}}\PY{p}{,} \PY{n}{y}\PY{o}{=}\PY{l+s+s2}{\PYZdq{}}\PY{l+s+s2}{value}\PY{l+s+s2}{\PYZdq{}}\PY{p}{)}\PY{p}{)}
    \PY{o}{+} \PY{n}{geom\PYZus{}boxplot}\PY{p}{(}\PY{n}{aes}\PY{p}{(}\PY{n}{fill}\PY{o}{=}\PY{l+s+s2}{\PYZdq{}}\PY{l+s+s2}{variable}\PY{l+s+s2}{\PYZdq{}}\PY{p}{)}\PY{p}{)}
    \PY{o}{+} \PY{n}{xlab}\PY{p}{(}\PY{l+s+s2}{\PYZdq{}}\PY{l+s+s2}{tourism type}\PY{l+s+s2}{\PYZdq{}}\PY{p}{)}
    \PY{o}{+} \PY{n}{ylab}\PY{p}{(}\PY{l+s+s2}{\PYZdq{}}\PY{l+s+s2}{tourists (x1000)}\PY{l+s+s2}{\PYZdq{}}\PY{p}{)}
    \PY{o}{+} \PY{n}{scale\PYZus{}y\PYZus{}continuous}\PY{p}{(}\PY{n}{breaks}\PY{o}{=}\PY{n}{np}\PY{o}{.}\PY{n}{arange}\PY{p}{(}\PY{l+m+mi}{0}\PY{p}{,} \PY{l+m+mi}{5000}\PY{p}{,} \PY{l+m+mi}{1000}\PY{p}{)}\PY{p}{,} 
                         \PY{n}{limits}\PY{o}{=}\PY{p}{[}\PY{l+m+mi}{0}\PY{p}{,} \PY{l+m+mi}{5000}\PY{p}{]}\PY{p}{)}
    \PY{o}{+} \PY{n}{scale\PYZus{}x\PYZus{}discrete}\PY{p}{(}\PY{n}{labels}\PY{o}{=}\PY{p}{[}\PY{l+s+s1}{\PYZsq{}}\PY{l+s+s1}{EF}\PY{l+s+s1}{\PYZsq{}}\PY{p}{,}\PY{l+s+s1}{\PYZsq{}}\PY{l+s+s1}{PC}\PY{l+s+s1}{\PYZsq{}}\PY{p}{,}\PY{l+s+s1}{\PYZsq{}}\PY{l+s+s1}{TI}\PY{l+s+s1}{\PYZsq{}}\PY{p}{,}\PY{l+s+s1}{\PYZsq{}}\PY{l+s+s1}{TF}\PY{l+s+s1}{\PYZsq{}}\PY{p}{]}\PY{p}{)}
    \PY{o}{+} \PY{n}{ggtitle}\PY{p}{(}\PY{l+s+s2}{\PYZdq{}}\PY{l+s+s2}{Tourists per type}\PY{l+s+s2}{\PYZdq{}}\PY{p}{)}
    \PY{o}{+} \PY{n}{theme}\PY{p}{(}\PY{n}{figure\PYZus{}size}\PY{o}{=}\PY{p}{(}\PY{l+m+mi}{6}\PY{p}{,}\PY{l+m+mi}{6}\PY{p}{)}\PY{p}{,}\PY{n}{legend\PYZus{}position}\PY{o}{=}\PY{l+s+s2}{\PYZdq{}}\PY{l+s+s2}{bottom}\PY{l+s+s2}{\PYZdq{}}\PY{p}{,}\PY{n}{legend\PYZus{}box\PYZus{}spacing}\PY{o}{=}\PY{l+m+mf}{0.4}\PY{p}{)}
    \PY{o}{+} \PY{n}{stat\PYZus{}summary}\PY{p}{(}\PY{n}{fun\PYZus{}y}\PY{o}{=} \PY{n}{mean}\PY{p}{,} \PY{n}{geom}\PY{o}{=}\PY{l+s+s2}{\PYZdq{}}\PY{l+s+s2}{point}\PY{l+s+s2}{\PYZdq{}}\PY{p}{,} \PY{n}{colour}\PY{o}{=}\PY{l+s+s2}{\PYZdq{}}\PY{l+s+s2}{black}\PY{l+s+s2}{\PYZdq{}}\PY{p}{,} \PY{n}{size}\PY{o}{=}\PY{l+m+mi}{2}\PY{p}{,}
               \PY{n}{position} \PY{o}{=} \PY{n}{position\PYZus{}dodge2}\PY{p}{(}\PY{n}{width} \PY{o}{=} \PY{l+m+mf}{0.75}\PY{p}{)}\PY{p}{)}
    \PY{o}{+}\PY{n}{guides}\PY{p}{(}\PY{n}{fill}\PY{o}{=}\PY{n}{guide\PYZus{}legend}\PY{p}{(}\PY{n}{title}\PY{o}{=}\PY{l+s+s2}{\PYZdq{}}\PY{l+s+s2}{tourism type}\PY{l+s+s2}{\PYZdq{}}\PY{p}{)}\PY{p}{)}
    \PY{o}{+}\PY{n}{scale\PYZus{}fill\PYZus{}manual}\PY{p}{(}\PY{n}{values}\PY{o}{=}\PY{n}{colors\PYZus{}palette}\PY{p}{,}\PY{n}{labels}\PY{o}{=}\PY{p}{[}\PY{l+s+s1}{\PYZsq{}}\PY{l+s+s1}{EF}\PY{l+s+s1}{\PYZsq{}}\PY{p}{,}\PY{l+s+s1}{\PYZsq{}}\PY{l+s+s1}{PC}\PY{l+s+s1}{\PYZsq{}}\PY{p}{,}\PY{l+s+s1}{\PYZsq{}}\PY{l+s+s1}{TI}\PY{l+s+s1}{\PYZsq{}}\PY{p}{,}\PY{l+s+s1}{\PYZsq{}}\PY{l+s+s1}{TF}\PY{l+s+s1}{\PYZsq{}}\PY{p}{]}\PY{p}{)}
    \PY{o}{+}\PY{n}{annotate}\PY{p}{(}\PY{n}{geom}\PY{o}{=}\PY{l+s+s1}{\PYZsq{}}\PY{l+s+s1}{text}\PY{l+s+s1}{\PYZsq{}}\PY{p}{,}\PY{n}{x}\PY{o}{=}\PY{l+m+mi}{1}\PY{p}{,}\PY{n}{y}\PY{o}{=}\PY{n}{np}\PY{o}{.}\PY{n}{floor}\PY{p}{(}\PY{n}{means}\PY{o}{.}\PY{n}{loc}\PY{p}{[}\PY{l+s+s1}{\PYZsq{}}\PY{l+s+s1}{excursionistas \PYZus{}fronterizos}\PY{l+s+s1}{\PYZsq{}}\PY{p}{]}\PY{p}{[}\PY{l+m+mi}{0}\PY{p}{]}\PY{p}{)}\PY{o}{\PYZhy{}}\PY{l+m+mi}{200}\PY{p}{,}\PY{n}{label}\PY{o}{=}\PY{n}{np}\PY{o}{.}\PY{n}{floor}\PY{p}{(}\PY{n}{means}\PY{o}{.}\PY{n}{loc}\PY{p}{[}\PY{l+s+s1}{\PYZsq{}}\PY{l+s+s1}{excursionistas \PYZus{}fronterizos}\PY{l+s+s1}{\PYZsq{}}\PY{p}{]}\PY{p}{[}\PY{l+m+mi}{0}\PY{p}{]}\PY{p}{)}\PY{p}{,}\PY{n}{size} \PY{o}{=} \PY{l+m+mi}{8}\PY{p}{)}
    \PY{o}{+}\PY{n}{annotate}\PY{p}{(}\PY{n}{geom}\PY{o}{=}\PY{l+s+s1}{\PYZsq{}}\PY{l+s+s1}{text}\PY{l+s+s1}{\PYZsq{}}\PY{p}{,}\PY{n}{x}\PY{o}{=}\PY{l+m+mi}{2}\PY{p}{,}\PY{n}{y}\PY{o}{=}\PY{n}{np}\PY{o}{.}\PY{n}{floor}\PY{p}{(}\PY{n}{means}\PY{o}{.}\PY{n}{loc}\PY{p}{[}\PY{l+s+s1}{\PYZsq{}}\PY{l+s+s1}{pasajeros\PYZus{}crucero}\PY{l+s+s1}{\PYZsq{}}\PY{p}{]}\PY{p}{[}\PY{l+m+mi}{0}\PY{p}{]}\PY{p}{)}\PY{o}{\PYZhy{}}\PY{l+m+mi}{200}\PY{p}{,}\PY{n}{label}\PY{o}{=}\PY{n}{np}\PY{o}{.}\PY{n}{floor}\PY{p}{(}\PY{n}{means}\PY{o}{.}\PY{n}{loc}\PY{p}{[}\PY{l+s+s1}{\PYZsq{}}\PY{l+s+s1}{pasajeros\PYZus{}crucero}\PY{l+s+s1}{\PYZsq{}}\PY{p}{]}\PY{p}{[}\PY{l+m+mi}{0}\PY{p}{]}\PY{p}{)}\PY{p}{,}\PY{n}{size} \PY{o}{=} \PY{l+m+mi}{8}\PY{p}{)}
    \PY{o}{+}\PY{n}{annotate}\PY{p}{(}\PY{n}{geom}\PY{o}{=}\PY{l+s+s1}{\PYZsq{}}\PY{l+s+s1}{text}\PY{l+s+s1}{\PYZsq{}}\PY{p}{,}\PY{n}{x}\PY{o}{=}\PY{l+m+mi}{3}\PY{p}{,}\PY{n}{y}\PY{o}{=}\PY{n}{np}\PY{o}{.}\PY{n}{floor}\PY{p}{(}\PY{n}{means}\PY{o}{.}\PY{n}{loc}\PY{p}{[}\PY{l+s+s1}{\PYZsq{}}\PY{l+s+s1}{turismo\PYZus{}al\PYZus{}interior}\PY{l+s+s1}{\PYZsq{}}\PY{p}{]}\PY{p}{[}\PY{l+m+mi}{0}\PY{p}{]}\PY{p}{)}\PY{o}{\PYZhy{}}\PY{l+m+mi}{200}\PY{p}{,}\PY{n}{label}\PY{o}{=}\PY{n}{np}\PY{o}{.}\PY{n}{floor}\PY{p}{(}\PY{n}{means}\PY{o}{.}\PY{n}{loc}\PY{p}{[}\PY{l+s+s1}{\PYZsq{}}\PY{l+s+s1}{turismo\PYZus{}al\PYZus{}interior}\PY{l+s+s1}{\PYZsq{}}\PY{p}{]}\PY{p}{[}\PY{l+m+mi}{0}\PY{p}{]}\PY{p}{)}\PY{p}{,}\PY{n}{size} \PY{o}{=} \PY{l+m+mi}{8}\PY{p}{)}
    \PY{o}{+}\PY{n}{annotate}\PY{p}{(}\PY{n}{geom}\PY{o}{=}\PY{l+s+s1}{\PYZsq{}}\PY{l+s+s1}{text}\PY{l+s+s1}{\PYZsq{}}\PY{p}{,}\PY{n}{x}\PY{o}{=}\PY{l+m+mi}{4}\PY{p}{,}\PY{n}{y}\PY{o}{=}\PY{n}{np}\PY{o}{.}\PY{n}{floor}\PY{p}{(}\PY{n}{means}\PY{o}{.}\PY{n}{loc}\PY{p}{[}\PY{l+s+s1}{\PYZsq{}}\PY{l+s+s1}{turismo\PYZus{}fronterizo}\PY{l+s+s1}{\PYZsq{}}\PY{p}{]}\PY{p}{[}\PY{l+m+mi}{0}\PY{p}{]}\PY{p}{)}\PY{o}{\PYZhy{}}\PY{l+m+mi}{200}\PY{p}{,}\PY{n}{label}\PY{o}{=}\PY{n}{np}\PY{o}{.}\PY{n}{floor}\PY{p}{(}\PY{n}{means}\PY{o}{.}\PY{n}{loc}\PY{p}{[}\PY{l+s+s1}{\PYZsq{}}\PY{l+s+s1}{turismo\PYZus{}fronterizo}\PY{l+s+s1}{\PYZsq{}}\PY{p}{]}\PY{p}{[}\PY{l+m+mi}{0}\PY{p}{]}\PY{p}{)}\PY{p}{,}\PY{n}{size} \PY{o}{=} \PY{l+m+mi}{8}\PY{p}{)}
\PY{p}{)}
\PY{n}{gg3}
\end{Verbatim}
\end{tcolorbox}

    \begin{center}
    \adjustimage{max size={0.9\linewidth}{0.9\paperheight}}{Take-Home-test-Nu-Roberto_files/Take-Home-test-Nu-Roberto_42_0.png}
    \end{center}
    { \hspace*{\fill} \\}
    
            \begin{tcolorbox}[breakable, size=fbox, boxrule=.5pt, pad at break*=1mm, opacityfill=0]
\prompt{Out}{outcolor}{16}{\boxspacing}
\begin{Verbatim}[commandchars=\\\{\}]
<ggplot: (8754054257255)>
\end{Verbatim}
\end{tcolorbox}
        
    The last plot show outliers in \texttt{turismo\_al\_interior} and
asymmetry in \texttt{excursionistas\_fronterizos}. That could be explain
for the bigger drop in 2020 of this cathegories in a short period of
time with respect to the others. In addition, we observe a high
variability in the cathegory of \texttt{excursionistas\_fronterizos},
therefore we can draw from the data that this tourism type changed more
from 2016 to 2021. Moreover, we show in number the mean of each
cathegory during that period.

Finally, we are going to show the mean proportion of each tourism
cathegory per year.

    \begin{tcolorbox}[breakable, size=fbox, boxrule=1pt, pad at break*=1mm,colback=cellbackground, colframe=cellborder]
\prompt{In}{incolor}{17}{\boxspacing}
\begin{Verbatim}[commandchars=\\\{\}]
\PY{c+c1}{\PYZsh{} Import libraries}
\PY{k+kn}{from} \PY{n+nn}{matplotlib} \PY{k+kn}{import} \PY{n}{pyplot} \PY{k}{as} \PY{n}{plt}
\PY{n}{plt}\PY{o}{.}\PY{n}{style}\PY{o}{.}\PY{n}{use}\PY{p}{(}\PY{l+s+s1}{\PYZsq{}}\PY{l+s+s1}{ggplot}\PY{l+s+s1}{\PYZsq{}}\PY{p}{)}
\PY{n}{fig}\PY{p}{,}\PY{n}{axs}\PY{o}{=} \PY{n}{plt}\PY{o}{.}\PY{n}{subplots}\PY{p}{(}\PY{l+m+mi}{2}\PY{p}{,}\PY{l+m+mi}{3}\PY{p}{,}\PY{n}{figsize}\PY{o}{=}\PY{p}{(}\PY{l+m+mi}{15}\PY{p}{,}\PY{l+m+mi}{15}\PY{p}{)}\PY{p}{)}
\PY{n}{list\PYZus{}of\PYZus{}years}\PY{o}{=}\PY{n+nb}{list}\PY{p}{(}\PY{n}{set\PYZus{}years\PYZus{}tourism}\PY{p}{)}
\PY{n}{dic\PYZus{}tourism\PYZus{}cathegory}\PY{o}{=}\PY{p}{\PYZob{}}
\PY{l+s+s1}{\PYZsq{}}\PY{l+s+s1}{excursionistas \PYZus{}fronterizos}\PY{l+s+s1}{\PYZsq{}}\PY{p}{:}\PY{l+s+s1}{\PYZsq{}}\PY{l+s+s1}{EF}\PY{l+s+s1}{\PYZsq{}}\PY{p}{,}
\PY{l+s+s1}{\PYZsq{}}\PY{l+s+s1}{pasajeros\PYZus{}crucero}\PY{l+s+s1}{\PYZsq{}}\PY{p}{:}\PY{l+s+s1}{\PYZsq{}}\PY{l+s+s1}{PC}\PY{l+s+s1}{\PYZsq{}}\PY{p}{,}
\PY{l+s+s1}{\PYZsq{}}\PY{l+s+s1}{turismo\PYZus{}al\PYZus{}interior}\PY{l+s+s1}{\PYZsq{}}\PY{p}{:}\PY{l+s+s1}{\PYZsq{}}\PY{l+s+s1}{TI}\PY{l+s+s1}{\PYZsq{}}\PY{p}{,}
\PY{l+s+s1}{\PYZsq{}}\PY{l+s+s1}{turismo\PYZus{}fronterizo}\PY{l+s+s1}{\PYZsq{}}\PY{p}{:}\PY{l+s+s1}{\PYZsq{}}\PY{l+s+s1}{TF}\PY{l+s+s1}{\PYZsq{}}
\PY{p}{\PYZcb{}}
\PY{n}{dic\PYZus{}colors\PYZus{}cathegory}\PY{o}{=}\PY{p}{\PYZob{}}
\PY{l+s+s1}{\PYZsq{}}\PY{l+s+s1}{excursionistas \PYZus{}fronterizos}\PY{l+s+s1}{\PYZsq{}}\PY{p}{:}\PY{n}{colors\PYZus{}palette}\PY{p}{[}\PY{l+m+mi}{0}\PY{p}{]}\PY{p}{,}
\PY{l+s+s1}{\PYZsq{}}\PY{l+s+s1}{pasajeros\PYZus{}crucero}\PY{l+s+s1}{\PYZsq{}}\PY{p}{:}\PY{n}{colors\PYZus{}palette}\PY{p}{[}\PY{l+m+mi}{1}\PY{p}{]}\PY{p}{,}
\PY{l+s+s1}{\PYZsq{}}\PY{l+s+s1}{turismo\PYZus{}al\PYZus{}interior}\PY{l+s+s1}{\PYZsq{}}\PY{p}{:}\PY{n}{colors\PYZus{}palette}\PY{p}{[}\PY{l+m+mi}{2}\PY{p}{]}\PY{p}{,}
\PY{l+s+s1}{\PYZsq{}}\PY{l+s+s1}{turismo\PYZus{}fronterizo}\PY{l+s+s1}{\PYZsq{}}\PY{p}{:}\PY{n}{colors\PYZus{}palette}\PY{p}{[}\PY{l+m+mi}{3}\PY{p}{]}
\PY{p}{\PYZcb{}}
\PY{n}{index\PYZus{}year}\PY{o}{=}\PY{l+m+mi}{0}
\PY{k}{def} \PY{n+nf}{make\PYZus{}autopct}\PY{p}{(}\PY{n}{values}\PY{p}{)}\PY{p}{:}
    \PY{k}{def} \PY{n+nf}{my\PYZus{}autopct}\PY{p}{(}\PY{n}{pct}\PY{p}{)}\PY{p}{:}
        \PY{n}{total} \PY{o}{=} \PY{n+nb}{sum}\PY{p}{(}\PY{n}{values}\PY{p}{)}
        \PY{n}{val} \PY{o}{=} \PY{n+nb}{int}\PY{p}{(}\PY{n+nb}{round}\PY{p}{(}\PY{n}{pct}\PY{o}{*}\PY{n}{total}\PY{o}{/}\PY{l+m+mf}{100.0}\PY{p}{)}\PY{p}{)}
        \PY{k}{return} \PY{l+s+s1}{\PYZsq{}}\PY{l+s+si}{\PYZob{}p:.2f\PYZcb{}}\PY{l+s+s1}{\PYZpc{}}\PY{l+s+s1}{ }\PY{l+s+se}{\PYZbs{}n}\PY{l+s+s1}{ (}\PY{l+s+si}{\PYZob{}v:d\PYZcb{}}\PY{l+s+s1}{)}\PY{l+s+s1}{\PYZsq{}}\PY{o}{.}\PY{n}{format}\PY{p}{(}\PY{n}{p}\PY{o}{=}\PY{n}{pct}\PY{p}{,}\PY{n}{v}\PY{o}{=}\PY{n}{val}\PY{p}{)}
    \PY{k}{return} \PY{n}{my\PYZus{}autopct}
\PY{k}{for} \PY{n}{i} \PY{o+ow}{in} \PY{n+nb}{range}\PY{p}{(}\PY{l+m+mi}{2}\PY{p}{)}\PY{p}{:}
    \PY{k}{for} \PY{n}{j} \PY{o+ow}{in} \PY{n+nb}{range}\PY{p}{(}\PY{l+m+mi}{3}\PY{p}{)}\PY{p}{:}
        \PY{n}{df\PYZus{}aux}\PY{o}{=}\PY{n}{mexico\PYZus{}tourism}\PY{p}{[}\PY{n}{mexico\PYZus{}tourism}\PY{p}{[}\PY{l+s+s1}{\PYZsq{}}\PY{l+s+s1}{year}\PY{l+s+s1}{\PYZsq{}}\PY{p}{]}\PY{o}{==}\PY{n}{list\PYZus{}of\PYZus{}years}\PY{p}{[}\PY{n}{index\PYZus{}year}\PY{p}{]}\PY{p}{]}
        \PY{n}{total\PYZus{}tourists}\PY{o}{=}\PY{n}{df\PYZus{}aux}\PY{p}{[}\PY{l+s+s1}{\PYZsq{}}\PY{l+s+s1}{visitantes\PYZus{}internacionales}\PY{l+s+s1}{\PYZsq{}}\PY{p}{]}\PY{o}{.}\PY{n}{to\PYZus{}numpy}\PY{p}{(}\PY{p}{)}
        \PY{n}{mean\PYZus{}total\PYZus{}tourists}\PY{o}{=}\PY{n}{np}\PY{o}{.}\PY{n}{mean}\PY{p}{(}\PY{n}{total\PYZus{}tourists}\PY{p}{)}
        \PY{n}{df\PYZus{}aux}\PY{o}{=}\PY{n}{df\PYZus{}aux}\PY{p}{[}\PY{n}{tourism\PYZus{}cathegory}\PY{p}{]}
        \PY{n}{cathegory\PYZus{}mean}\PY{o}{=}\PY{n}{pd}\PY{o}{.}\PY{n}{DataFrame}\PY{p}{(}\PY{n}{df\PYZus{}aux}\PY{o}{.}\PY{n}{mean}\PY{p}{(}\PY{p}{)}\PY{p}{)}
        \PY{n}{pie\PYZus{}labels}\PY{o}{=}\PY{p}{[}\PY{n}{dic\PYZus{}tourism\PYZus{}cathegory}\PY{p}{[}\PY{n}{c}\PY{p}{]} \PY{k}{for} \PY{n}{c} \PY{o+ow}{in} \PY{n}{cathegory\PYZus{}mean}\PY{o}{.}\PY{n}{index}\PY{o}{.}\PY{n}{to\PYZus{}list}\PY{p}{(}\PY{p}{)}\PY{p}{]}
        \PY{n}{pie\PYZus{}colors}\PY{o}{=}\PY{p}{[}\PY{n}{dic\PYZus{}colors\PYZus{}cathegory}\PY{p}{[}\PY{n}{c}\PY{p}{]} \PY{k}{for} \PY{n}{c} \PY{o+ow}{in} \PY{n}{cathegory\PYZus{}mean}\PY{o}{.}\PY{n}{index}\PY{o}{.}\PY{n}{to\PYZus{}list}\PY{p}{(}\PY{p}{)}\PY{p}{]}
        \PY{n}{wedges}\PY{p}{,} \PY{n}{texts}\PY{p}{,} \PY{n}{autotexts}\PY{o}{=}\PY{n}{axs}\PY{p}{[}\PY{n}{i}\PY{p}{,}\PY{n}{j}\PY{p}{]}\PY{o}{.}\PY{n}{pie}\PY{p}{(}\PY{n}{cathegory\PYZus{}mean}\PY{p}{[}\PY{l+m+mi}{0}\PY{p}{]}\PY{o}{.}\PY{n}{to\PYZus{}numpy}\PY{p}{(}\PY{p}{)}\PY{p}{,}
        \PY{n}{autopct} \PY{o}{=} \PY{n}{make\PYZus{}autopct}\PY{p}{(}\PY{n}{cathegory\PYZus{}mean}\PY{p}{[}\PY{l+m+mi}{0}\PY{p}{]}\PY{o}{.}\PY{n}{to\PYZus{}numpy}\PY{p}{(}\PY{p}{)}\PY{p}{)}\PY{p}{,}
        \PY{n}{shadow} \PY{o}{=} \PY{k+kc}{True}\PY{p}{,}
        \PY{n}{labels}\PY{o}{=}\PY{n}{pie\PYZus{}labels}\PY{p}{,}
        \PY{n}{colors}\PY{o}{=}\PY{n}{pie\PYZus{}colors}\PY{p}{,}
        \PY{n}{startangle} \PY{o}{=} \PY{l+m+mi}{90}\PY{p}{,}
        \PY{n}{textprops}\PY{o}{=}\PY{n+nb}{dict}\PY{p}{(}\PY{n}{color}\PY{o}{=}\PY{l+s+s1}{\PYZsq{}}\PY{l+s+s1}{k}\PY{l+s+s1}{\PYZsq{}}\PY{p}{,} \PY{n}{fontsize}\PY{o}{=}\PY{l+m+mi}{10}\PY{p}{)}\PY{p}{,}
        \PY{n}{radius}\PY{o}{=}\PY{l+m+mf}{1.2}
        \PY{p}{)}
        \PY{n}{axs}\PY{p}{[}\PY{n}{i}\PY{p}{,}\PY{n}{j}\PY{p}{]}\PY{o}{.}\PY{n}{set\PYZus{}title}\PY{p}{(}\PY{n}{list\PYZus{}of\PYZus{}years}\PY{p}{[}\PY{n}{index\PYZus{}year}\PY{p}{]}\PY{p}{,}\PY{n}{pad}\PY{o}{=}\PY{l+m+mi}{20}\PY{p}{)}
        \PY{n}{index\PYZus{}year}\PY{o}{+}\PY{o}{=}\PY{l+m+mi}{1}
\PY{c+c1}{\PYZsh{} Adding legend}
\PY{n}{axs}\PY{p}{[}\PY{n}{i}\PY{p}{,}\PY{n}{j}\PY{p}{]}\PY{o}{.}\PY{n}{legend}\PY{p}{(}\PY{n}{wedges}\PY{p}{,} \PY{n}{pie\PYZus{}labels}\PY{p}{,}
    \PY{n}{title} \PY{o}{=}\PY{l+s+s2}{\PYZdq{}}\PY{l+s+s2}{tourism type}\PY{l+s+s2}{\PYZdq{}}\PY{p}{,}
    \PY{n}{loc} \PY{o}{=}\PY{l+s+s2}{\PYZdq{}}\PY{l+s+s2}{center left}\PY{l+s+s2}{\PYZdq{}}\PY{p}{,}
    \PY{n}{bbox\PYZus{}to\PYZus{}anchor} \PY{o}{=}\PY{p}{(}\PY{l+m+mf}{1.2}\PY{p}{,} \PY{l+m+mi}{0}\PY{p}{,} \PY{l+m+mf}{0.6}\PY{p}{,} \PY{l+m+mi}{1}\PY{p}{)}\PY{p}{)}
\PY{n}{plt}\PY{o}{.}\PY{n}{subplots\PYZus{}adjust}\PY{p}{(}\PY{n}{left}\PY{o}{=}\PY{l+m+mf}{0.1}\PY{p}{,}
                    \PY{n}{bottom}\PY{o}{=}\PY{l+m+mf}{0.2}\PY{p}{,} 
                    \PY{n}{right}\PY{o}{=}\PY{l+m+mf}{0.9}\PY{p}{,} 
                    \PY{n}{top}\PY{o}{=}\PY{l+m+mf}{0.7}\PY{p}{,} 
                    \PY{n}{wspace}\PY{o}{=}\PY{l+m+mf}{0.3}\PY{p}{,} 
                    \PY{n}{hspace}\PY{o}{=}\PY{l+m+mf}{0.3}\PY{p}{)}
\end{Verbatim}
\end{tcolorbox}

    \begin{center}
    \adjustimage{max size={0.9\linewidth}{0.9\paperheight}}{Take-Home-test-Nu-Roberto_files/Take-Home-test-Nu-Roberto_44_0.png}
    \end{center}
    { \hspace*{\fill} \\}
    
    In this plot, we show the mean proportion of each tourism cathegory per
year. We observe an increasing behavior of
\texttt{turismo\ al\ interior} with respect to the mean of each year.
Analogously, we observe that in \texttt{turismo\ fronterizo}, with the
only exception of 2020 where every cathegory dropped. On the other hand,
we observe an decreasing behavior from 2016 to 2022 of
\texttt{pasajeros\ crucero} and \texttt{excursionistas\ fronterizos}.

We have already seen that each cathegory drop from 2019 to 2020.
Computing the relative drop from 2019 to 2020 of each cathegory we
obtain that \texttt{pasajeros\ crucero} dropped to 28\% of tourists with
respect to 2019,\texttt{turismo\ al\ interior} dropped to 45\%,
\texttt{excursionistas\ fronterizos} dropped to 59\%, and
\texttt{turismo\ fronterizo} dropped to 63\% with respect to 2019's
data. We can conclude that \texttt{pasajeros\ crucero} was the cathegory
with the higher drop with respect to the mean in 2020.

    \hypertarget{creativity-to-communicate-analytical-results}{%
\subsubsection{Creativity to communicate analytical
results}\label{creativity-to-communicate-analytical-results}}

    Finally, we will answer The Leadership team's question: \emph{The number
of Pueblos mágicos in each state}

It is an interesting question know if there are states that do not have
\emph{Pueblos mágicos}, therefore a barplot is natural option for that
task because we can see and contrast the number of \emph{Pueblos
mágicos} of each Mexico state. However, in this plot we have 32 states
and it could be difficult to see the exact number of \emph{Pueblos
mágicos} and how many states have the same number of \emph{Pueblos
mágicos}.

In the following code cell we will show the plot that tries to solve
those problems and with the information required by The Leadership team.

    \begin{tcolorbox}[breakable, size=fbox, boxrule=1pt, pad at break*=1mm,colback=cellbackground, colframe=cellborder]
\prompt{In}{incolor}{18}{\boxspacing}
\begin{Verbatim}[commandchars=\\\{\}]
\PY{n}{dic\PYZus{}num\PYZus{}magic\PYZus{}towns\PYZus{}per\PYZus{}state}\PY{o}{=}\PY{p}{\PYZob{}}\PY{p}{\PYZcb{}}
\PY{n}{estado\PYZus{}iso\PYZus{}code}\PY{o}{=}\PY{n}{df\PYZus{}estado\PYZus{}iso}\PY{p}{[}\PY{l+s+s1}{\PYZsq{}}\PY{l+s+s1}{3\PYZhy{}letter\PYZhy{}code}\PY{l+s+s1}{\PYZsq{}}\PY{p}{]}\PY{o}{.}\PY{n}{to\PYZus{}list}\PY{p}{(}\PY{p}{)}
\PY{k}{for} \PY{n}{state} \PY{o+ow}{in} \PY{n}{estado\PYZus{}iso\PYZus{}code}\PY{p}{:}
    \PY{n}{df\PYZus{}aux}\PY{o}{=}\PY{n}{magic\PYZus{}towns}\PY{p}{[}\PY{n}{magic\PYZus{}towns}\PY{p}{[}\PY{l+s+s1}{\PYZsq{}}\PY{l+s+s1}{estado}\PY{l+s+s1}{\PYZsq{}}\PY{p}{]}\PY{o}{==}\PY{n}{state}\PY{p}{]}
    \PY{n}{dic\PYZus{}num\PYZus{}magic\PYZus{}towns\PYZus{}per\PYZus{}state}\PY{p}{[}\PY{n}{state}\PY{p}{]}\PY{o}{=}\PY{n}{df\PYZus{}aux}\PY{o}{.}\PY{n}{shape}\PY{p}{[}\PY{l+m+mi}{0}\PY{p}{]}
\PY{n}{df\PYZus{}num\PYZus{}magic\PYZus{}towns\PYZus{}per\PYZus{}state}\PY{o}{=}\PY{n}{pd}\PY{o}{.}\PY{n}{DataFrame}\PY{p}{(}\PY{p}{\PYZob{}}\PY{l+s+s1}{\PYZsq{}}\PY{l+s+s1}{estado}\PY{l+s+s1}{\PYZsq{}}\PY{p}{:}\PY{n}{dic\PYZus{}num\PYZus{}magic\PYZus{}towns\PYZus{}per\PYZus{}state}\PY{o}{.}\PY{n}{keys}\PY{p}{(}\PY{p}{)}\PY{p}{,}\PY{l+s+s1}{\PYZsq{}}\PY{l+s+s1}{num\PYZus{}pueblos\PYZus{}magicos}\PY{l+s+s1}{\PYZsq{}}\PY{p}{:}\PY{n}{dic\PYZus{}num\PYZus{}magic\PYZus{}towns\PYZus{}per\PYZus{}state}\PY{o}{.}\PY{n}{values}\PY{p}{(}\PY{p}{)}\PY{p}{\PYZcb{}}\PY{p}{)}
\PY{n}{gg5}\PY{o}{=}\PY{p}{(}
    \PY{n}{ggplot}\PY{p}{(}\PY{n}{df\PYZus{}num\PYZus{}magic\PYZus{}towns\PYZus{}per\PYZus{}state}\PY{p}{,}\PY{n}{aes}\PY{p}{(}\PY{n}{x}\PY{o}{=}\PY{l+s+s1}{\PYZsq{}}\PY{l+s+s1}{estado}\PY{l+s+s1}{\PYZsq{}}\PY{p}{,}\PY{n}{y}\PY{o}{=}\PY{l+s+s1}{\PYZsq{}}\PY{l+s+s1}{num\PYZus{}pueblos\PYZus{}magicos}\PY{l+s+s1}{\PYZsq{}}\PY{p}{)}\PY{p}{)}
    \PY{o}{+}\PY{n}{geom\PYZus{}point}\PY{p}{(}\PY{n}{aes}\PY{p}{(}\PY{n}{size}\PY{o}{=}\PY{l+s+s1}{\PYZsq{}}\PY{l+s+s1}{num\PYZus{}pueblos\PYZus{}magicos}\PY{l+s+s1}{\PYZsq{}}\PY{p}{,}\PY{n}{color}\PY{o}{=}\PY{l+s+s1}{\PYZsq{}}\PY{l+s+s1}{factor(num\PYZus{}pueblos\PYZus{}magicos)}\PY{l+s+s1}{\PYZsq{}}\PY{p}{)}\PY{p}{)}
    \PY{o}{+}\PY{n}{scale\PYZus{}size\PYZus{}continuous}\PY{p}{(}\PY{n}{guide}\PY{o}{=}\PY{k+kc}{False}\PY{p}{,}\PY{n+nb}{range}\PY{o}{=}\PY{p}{(}\PY{l+m+mi}{5}\PY{p}{,}\PY{l+m+mi}{14}\PY{p}{)}\PY{p}{)}
    \PY{o}{+}\PY{n}{scale\PYZus{}color\PYZus{}brewer}\PY{p}{(}\PY{n+nb}{type}\PY{o}{=}\PY{l+s+s1}{\PYZsq{}}\PY{l+s+s1}{qual}\PY{l+s+s1}{\PYZsq{}}\PY{p}{,}\PY{n}{palette}\PY{o}{=}\PY{l+m+mi}{3}\PY{p}{,}\PY{n}{name}\PY{o}{=}\PY{l+s+s1}{\PYZsq{}}\PY{l+s+s1}{numero de pueblos}\PY{l+s+s1}{\PYZsq{}}\PY{p}{)}
    \PY{o}{+}\PY{n}{scale\PYZus{}x\PYZus{}discrete}\PY{p}{(}\PY{n}{expand}\PY{o}{=}\PY{p}{(}\PY{l+m+mi}{0}\PY{p}{,} \PY{l+m+mi}{1}\PY{p}{)}\PY{p}{)}
    \PY{o}{+}\PY{n}{labs}\PY{p}{(}\PY{n}{x}\PY{o}{=}\PY{l+s+s1}{\PYZsq{}}\PY{l+s+s1}{estado}\PY{l+s+s1}{\PYZsq{}}\PY{p}{,}\PY{n}{y}\PY{o}{=}\PY{l+s+s1}{\PYZsq{}}\PY{l+s+s1}{num de pueblos mágicos}\PY{l+s+s1}{\PYZsq{}}\PY{p}{)}
    \PY{o}{+}\PY{n}{scale\PYZus{}y\PYZus{}continuous}\PY{p}{(}\PY{n}{breaks}\PY{o}{=}\PY{n}{np}\PY{o}{.}\PY{n}{arange}\PY{p}{(}\PY{l+m+mi}{1}\PY{p}{,}\PY{l+m+mi}{11}\PY{p}{,}\PY{l+m+mi}{1}\PY{p}{)}\PY{p}{)}
    \PY{o}{+}\PY{n}{coord\PYZus{}flip}\PY{p}{(}\PY{p}{)}
    \PY{o}{+}\PY{n}{theme\PYZus{}light}\PY{p}{(}\PY{p}{)}
    \PY{o}{+}\PY{n}{theme}\PY{p}{(}\PY{n}{figure\PYZus{}size}\PY{o}{=}\PY{p}{(}\PY{l+m+mi}{8}\PY{p}{,}\PY{l+m+mi}{8}\PY{p}{)}\PY{p}{,}\PY{n}{legend\PYZus{}position}\PY{o}{=}\PY{l+s+s2}{\PYZdq{}}\PY{l+s+s2}{bottom}\PY{l+s+s2}{\PYZdq{}}\PY{p}{,}\PY{n}{legend\PYZus{}box\PYZus{}spacing}\PY{o}{=}\PY{l+m+mf}{0.5}\PY{p}{,}\PY{n}{line}\PY{o}{=}\PY{n}{element\PYZus{}line}\PY{p}{(}\PY{n}{color}\PY{o}{=}\PY{l+s+s1}{\PYZsq{}}\PY{l+s+s1}{black}\PY{l+s+s1}{\PYZsq{}}\PY{p}{)}\PY{p}{,}
    \PY{n}{legend\PYZus{}title\PYZus{}align}\PY{o}{=}\PY{l+s+s1}{\PYZsq{}}\PY{l+s+s1}{center}\PY{l+s+s1}{\PYZsq{}}\PY{p}{)}
\PY{p}{)}
\PY{n}{gg5}
\end{Verbatim}
\end{tcolorbox}

    \begin{center}
    \adjustimage{max size={0.9\linewidth}{0.9\paperheight}}{Take-Home-test-Nu-Roberto_files/Take-Home-test-Nu-Roberto_48_0.png}
    \end{center}
    { \hspace*{\fill} \\}
    
            \begin{tcolorbox}[breakable, size=fbox, boxrule=.5pt, pad at break*=1mm, opacityfill=0]
\prompt{Out}{outcolor}{18}{\boxspacing}
\begin{Verbatim}[commandchars=\\\{\}]
<ggplot: (8754084162482)>
\end{Verbatim}
\end{tcolorbox}
        
    The last graphs in x-axis has the number of \emph{Pueblos mágicos} and
in y-axis the states of Mexico. For each row or state we have a dot with
center aligned with the exact number \emph{Pueblos mágicos} in that
state, therefore we answer the main question. In addition, we can
observe that points with the same color correspond to states with the
same number of \emph{Pueblos mágicos} and the legend says what is the
number of \emph{Pueblos mágicos} these states have. Moreover, the size
of the dots increase with the number of \emph{Pueblos mágicos} that
represent. The increasing size helps to identify how many states have
the maximum and minimun number of \emph{Pueblos mágicos}, then you can
identify what are these states.

    \hypertarget{machine-learning-solutions-mindset}{%
\subsection{Machine Learning solutions
mindset}\label{machine-learning-solutions-mindset}}

    The \textbf{Business use case} we chose is \textbf{Case 1}. We propose
a solution for accepting credit card customers in a digital bank.

    \hypertarget{step-1}{%
\subsubsection{Step 1}\label{step-1}}

First of all, we think in the database which help us to make a decision.

In general, credit card decisions are made based in the credit history.
The credit history we will use is the \emph{Buró de crédito}'s history.

\emph{Buró de crédito}'s credit history for each person contain a list
in detail of past credits requested to other credit institutions and the
status of each of this credits.

There are 3 status in \emph{Buró de crédito} system, they are: 1. up to
date with payment. 2. From 1 to 89 delay days. 3. More than 90 delay
days.

Of course, the status depends on the type of credit, amount to pay,
credit limit, maximum credit, current balance and customer's current
income, etc.

We only have in mind the maximum credit, current balance, and the status
of all credits in the history because we are in a digital bank context.
That information could help to measure the payment ability of the
customer.

In \emph{Buró de crédito} we can find that input (maximum credit,current
balance) with its respect status.

With that information the idea is training a Neural Network for binary
classification problem. The classes are \emph{up to date with payment}
and the other one is if the status is \emph{From 1 to 89 delay days} or
\emph{More than 90 delay days}

The decision could be based in the output of this Neural Network given a
credit maximum and a fixed current balance.

    \hypertarget{step-2}{%
\subsubsection{Step 2}\label{step-2}}

In this part, we are going to state the last approach mathematically.

First of all, we need to abstract the data in mathematical notation.

The sample of a customer, obtaining in a Query to \emph{Buró de
crédito}'s database, is of the form
\[\{(X_{1i},X_{2i},Y_i)\in \mathbb{R}^2\ |\ Y_n\in\{0,1\},i=1,2,\dots,n\},\]

where \(n\) is the number of credits.

Here 1. \(X_{1i}\) is the maximum credit of \(i\)-th credit. 2.
\(X_{2i}\) is the current balance of \(i\)-th credit. 3. \(Y_i\) is a
dummy variable such as
\[Y_i=\left\lbrace\begin{matrix}1&\text{ if }i\text{-th credit has \textit{up to date payment} status}\\
0&\text{otherwise}\end{matrix}\right.\]

Determine the number of hidden layers and the number of neurons of each
layer is something that requires experimentation but it is task that can
be done with a framework.

Because of our approach our neural network has two neurons in the input
layer and one in the output layer.

We can use sigmoid activation function its form is
\[ \sigma(x)=\frac{1}{1+e^{-x}},\] and its range is \([0,1]\). Therefore
our Neural Network is a function \(f:\mathbb{R}^2\to [0,1]\).

For an input \((\hat{X}_1,\hat{X}_2)\) we set the correspondent dummy
variable as
\[\hat{Y}=\left\lbrace\begin{matrix}1&\text{ if }f(\hat{X}_1,\hat{X}_2)>0.5\\ 0&\text{ otherwise }\end{matrix}\right.\]

Thinking in a extreme case, if we want to accept credit card customer we
train the neural network with his personal dataset, then we test with
his required credit and a 0 or fixed tolerable current balance. We
accept the customer if we obtain a \(1\) in the last classification and
we do not accept him if we obtain a \(0\).

On the other hand, if we do not have enough data for the customer we can
fixed a credit based in his current income but it needs to be an amount
with a low risk for the company.

    \hypertarget{step-3}{%
\subsubsection{Step 3}\label{step-3}}

We need to identify the main computation tasks in this solution. The
first one is to obtain the personal data set of each customer from
\emph{Buró de Crédito}. The second one is the computational cost for
training the neural network in order to make a decision on the customer
request.

For the first task we need to get information from the customer and with
that information obtain the customer's sample stated in the last
section.

If we already have the customer's personal data, then we can ask for the
sample required for the neural network training through \emph{BigQuery}
by Google Cloud. This technology has support for connections to a
external database source. In addition, it allows to execute statistics
of large amount data in real-time that is exactly what we need in a
digital bank.

For the Machine Learning task, I mean, the Neural Network training, we
can use \emph{Vertex AI} by Google Cloud. This technology supports
AutoML that is a useful tool for training different Neural Networks
corresponding to each customer, and we avoid the hard task of checking
the hidden structure of the Neural Network. In addition,we can assess
the final model in a friendly way. Finally, we deploy the model for
online predictions required in our digital bank.


    % Add a bibliography block to the postdoc
    
    
    
\end{document}
